
given these differences and only relatively weak alignments in order to quantify the
degree of similarity we applied a... 

...method based on the generation of a population of 'decoy' models [50] which can
be compared with the SAP program [51] to provide a background distribution of scores.
This method has the advantage that each comparison in the random pool is between
two models of the same size and secondary structure composition as the pair of native structures
being investigated.
For this study we collected 10 capsid N-terminal domains and 7 C-terminal domains, each of which 
were compared with the foamy N-terminal domain and the foamy C-terminal domain.

The degree of similarity between the domains ranged from less than 2 sigma (Z-score)
to over 5 sigma, however, the latter (highly significant) result was obtained for both
a reversed (NC) and forward (CC) matching.  To obtain a more quantitative consensus
for the amino/amino (NN) and carboxy/carboxy (CC) versus the reversed domain pairings
(NC and CN), the raw results were combined, allowing the comparison of two distributions.
a significance was calculated using Student's T-test, the values of which are given
All four possible domain pairings were highly significant with probabilities ranging
from 10^10 to over 10^20.   However, two reversed pairings (NC and CN) have lower probabilities
than the forward pairings (NN and CC) and combining the probabilities 
gives a value of 18 logs difference which means that the reversed pairing is
very unlikely.
Therefor the preferred, and biologically more reasonable, result is that the ortho virus
domain are related to the foamy virus domains as a result of genetic divergence from
a common, double domain anscestor. 

The reversed pairings, nevertheless, have a high structural significance and this suggests
further that the two domains are derived from an earlier gene-duplication event that has
been retained more clearly in the foamy viruses.   Comparing the two foamy domins gives
a Z-score of 2.077 sigma which, although of marginal significance, supports this model.
A similar comparison in structures of the ortho virsuses gives a
similar picture on an individual basis, with Z-scores ranging from 2 to 4.  However,
as with the comparisons with the foamy virus, these can be pooled to allow a joint
T-test to be applied.   This gave a probability of 1/10^8 that the native N/C domain
comparisons were drawn from the random distribution, adding strong support to the
hypothesis of an ancient gene duplication occuring before the split of the ortho 
and foamy virus families.
