\paragraph{Background}
The {\em Spumaretrovirinae} (foamy viruses) and the {\em Orthoretrovirinae} (e.g. HIV) share
many similarities both in genome structure and the sequences of the core viral encoded proteins,
such as the aspartyl protease and reverse transcriptase.  Similarity in the {\em gag} region of the
genome is less obvious at the sequence level but has been illuminated by the recent solution of
the foamy virus capsid (CA) structure.   This revealed a clear structural similarity to the
orthoretrovirus capsids but with marked differences that left uncertainty in the relationship
between the two domains that comprise the structure.

\paragraph{Methods}
We have applied protein structure comparison methods to this problem in order to try and
resolve this ambiguity.  These included both the {\tt DALI} method and the {\tt SAP} method,
with with rigorous statistical tests applied to the results of both methods.  For this,
we employed collections of artificial fold 'decoys' (generated from the pair of native 
structures being compared) to provide a customised background distribution for each
comparison, thus allowing significance levels to be estimated.

\paragraph{Results}
We have shown that the relationship of the two domains
conforms to a simple linear correspondence rather than a domain transposition.   These
similarities suggest that the origin of both viral capsids was a common ancestor with a double
domain structure.  In addition, we show that there is also a significant structural similarity
between the amino and carboxy domains in both the foamy and ortho viruses which suggests that
there may have been an even more ancient gene-duplication that preceded the double domain structure.
In addition, our structure comparison methodology demonstrates a general approach to problems
where the components have a high intrinsic level of similarity.
