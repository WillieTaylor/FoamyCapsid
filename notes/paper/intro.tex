\section{Introduction}

Taxonomically, the {\em Orthoretrovirinae} (orthoretroviruses) and {\em Spumaretrovirinae}\footnote{
This class is also commonly referred to as the Foamy viruses (after the morphological effect they have on infected cells)
and will be referred by this name frequently below, with the term orthoretroviruses also contracted to "Ortho viruses".
} 
(spumaviruses) make up the two subfamilies of {\em Retroviridae}. They share many similarities, including overall genome
structures with gag, pol and env genes encoding proteins for replication and life cycles involving reverse transcription
and integration into the chromosomes of infected cells. However, there are also a number of differences distinguishing these
viral subfamilies, including finer details of genome organisation, the absence of a Gag-Pol fusion protein in spumaviruses
and the timing of reverse transcription.

Gag is the major structural protein of both Ortho and Foamy viruses and also displays both important differences and similarities.
Ortho and Foamy viral Gag are required for particle assembly, budding from the cell, reverse transcription and delivery of the
viral nucleic acid into the newly infected cell. However, there are a number of striking differences including how the Gag
precursor is targeted to the cell membrane, the absence of a Major Homology Region and Cys-His box in Foamy viruses and very 
different patterns of processing during viral maturation. In all Ortho viruses, Gag is proteolytically cleaved to form distinct, 
well-studied proteins, matrix (MA), capsid (CA) and nucleocapsid (NC), found in mature virions but in spumaviruses Gag processing 
does not occur. 

The recent solution of the Foamy Gag protein structure has shed new light on this relationship by revealing that
the capsid structures of both viral classes share a common protein fold, with the implication that their gag proteins may
be evolutionarily related \cite{BallNJet16}.   An intriguing aspect of this relationship was an ambiguity in the degree of
relatedness between the two domains of the gag proteins, with the Spumaretroviral Gag domains appearing almost equally
similar to both the amino- and carboxy-terminal domains of the orthoretroviruses.   In this paper, we investigate the
nature of this relationship in greater detail and discuss its evolutionary implications. 
