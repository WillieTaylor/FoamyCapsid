\section{Results}

\subsection{Full-length comparison}

To investigate the structural relationship between the capsid structure of the ortho viruses (HIV, MLV, etc.),
and the new structure of the foamy virus capsid \cite{Ian}, the foamy virus structure was compared to one
of the few full double domain ortho virus structures, the HIV capsid with PDB code: {\tt 3nte}, using the
flexible superposition program \SAP\ \cite{TaylorWR99a}.   Even though this program has a tolerant approach
to relative domain shifts, the comparison produced a high RMSD value of 14\AA\ over the 100 best superposed
positions.   The amino (N) terminal domain positions roughly corresponded but shifts in the relative
orientation of the carboxy (C) terminal domain resulted in large deviations between equivalent helices.  
The superposed structures are shown in \Fig{fullSAP} and the domain divergence can be seen clearly as a
jump in the cumulative RMSD plot (\Fig{fullRMS}).

\begin{figure}
\centering
\subfigure[]{
\label{Fig:fullSAP}
\epsfxsize=300pt \epsfbox{figs/full3nte.super.eps}
}
\subfigure[]{
\label{Fig:fullRMS}
\rotatebox{270}{
\epsfxsize=220pt \epsfbox{figs/plotrms.eps}
}}
\begin{footnotesize}
\caption{
\label{Fig:full}
{\bf Full ortho/foamy virus capsid superposition}.
The superposed structures are shown in part ($a$) as a stereo pair, coloured as green = ortho virus (HIV, PDB code:{\tt 3nte-A})
and magenta = foamy virus capsid.   The interior of the virus would lie below?. 
Part ($b$) shows the cumulative RMSD plot for this superposition which plots the RMSD value (Y-axis) for increasingly larger sets
of residues as ranked by their \SAP\ similarity score (X-axis).   The sharp rise in this trace marks the transition into 
subsets that include positions from the displaced domain.
}
\end{footnotesize}
\end{figure}

\subsection{DALI searches}

Although this initial superposition (\Fig{full}) did not appear encouraging, the foamy virus structure
was scanned across the Protein DataBank (PDB), using the \DALI\ program \cite{HolmLet93} to search for any similarities.

\subsubsection{Full chain scan}

A scan of the full-length foamy structure using the DALI server\footnote{
{\tt http://ekhidna.biocenter.helsinki.fi/dali\_server},
see Methods section for details.
}
over the 90\% non-redundant protein structure databank
identified a wide selection of retroviral capsid structures.  In the ranked list of structue hits,
capsids were identified from position 2 to position 550.
The top hits are shown in \Fig{dali} (See Supplementary material for a summary of
the full 550 with Z-scores over 2).    Although the top scoring hit is not a capsid
protein, many are found in the top 20 hits.   However, almost all of these are partial
hits, covering little more than half the query structure.   The structural alignment of the top two hits
is shown in \Fig{top2} coloured to emphasise the matched regions.

\begin{figure}
\centering
\begin{singlespace}
\begin{tiny}
\begin{Verbatim}[frame=single]
   No:  Chain   Z    rmsd lali nres  %id PDB  Description
    1:  4x3x-A  5.0  3.1   66    82   11 PDB  MOLECULE: ACTIVITY-REGULATED CYTOSKELETON-ASSOC
    2|  3g29-A  3.7  2.7   60    77    8 PDB  MOLECULE: GAG POLYPROTEIN;                                           
    3|  3g0v-A  3.7  2.9   62    76    8 PDB  MOLECULE: GAG POLYPROTEIN;                                           
    4:  2v50-D  3.6  2.2   41   998    7 PDB  MOLECULE: MULTIDRUG RESISTANCE PROTEIN MEXB;                         
    5:  3j39-i  3.6  2.5   40   113    3 PDB  MOLECULE: 60S RIBOSOMAL PROTEIN L10A-2;                              
    6|  4ph2-A  3.6  3.2   69   127    7 PDB  MOLECULE: BLV CAPSID - N-TERMINAL DOMAIN;                            
    7:  1iqp-E  3.6  3.8   69   326    7 PDB  MOLECULE: RFCS;                                                      
    8:  4gco-A  3.6  3.7   55   120   11 PDB  MOLECULE: PROTEIN STI-1;                                             
    9|  3g29-B  3.6  2.8   62    77    8 PDB  MOLECULE: GAG POLYPROTEIN;                                           
   10|  3g1i-B  3.6  2.9   62    75    8 PDB  MOLECULE: GAG POLYPROTEIN;                                           
   11|  3g21-A  3.6  2.8   60    77    8 PDB  MOLECULE: GAG POLYPROTEIN;                                           
   12:  2a0u-A  3.5  3.1   68   374    4 PDB  MOLECULE: INITIATION FACTOR 2B;                                      
   13:  1j7q-A  3.5  2.9   60    86    5 PDB  MOLECULE: CALCIUM VECTOR PROTEIN;                                    
   14:  2a0u-B  3.5  8.1   80   367    4 PDB  MOLECULE: INITIATION FACTOR 2B;                                      
   15:  1iqp-A  3.5  3.7   70   326    7 PDB  MOLECULE: RFCS;                                                      
   16|  4ph0-C  3.5  4.6  101   199    8 PDB  MOLECULE: BLV CAPSID;                                                
   17|  4ph0-D  3.5  4.2  101   198    8 PDB  MOLECULE: BLV CAPSID;                                                
   18|  4ph2-B  3.5  3.3   69   127    7 PDB  MOLECULE: BLV CAPSID - N-TERMINAL DOMAIN;                            
   19:  1sxj-B  3.4  3.5   65   316    3 PDB  MOLECULE: ACTIVATOR 1 95 KDA SUBUNIT;                                
   20:  2afd-A  3.4  2.7   59    88   14 PDB  MOLECULE: PROTEIN ASL1650;                                           
\end{Verbatim}
\end{tiny}
\end{singlespace}
\begin{footnotesize}
\caption{
\label{Fig:dali}
{\bf Top structural similarities}
found by the \DALI\ program in the 90\% non-redundant PDB (PDB-90) using the full length foamy
virus capsid as a query (145 residues).
The columns are: the ranked number of the hit ({\tt No.}), marked by a '{\tt |}' for a capsid protein, otherwise '{\tt :}';
the PDB entry identifier ({\tt Chain}, with the chain designation after the dash); the \DALI\ Z-score ({\tt Z})
(significance estimate); the root-mean-square-deviation ({\tt rmsd}) over aligned \CA\ positions; the number
of aligned positions ({\tt lali}); the number of residues in the matched structure ({\tt nres}); the percentage
sequence identity of the match ({\tt \%id}) followed by a description of the molecule.
It can be seen from the number of matched positions ({\tt lali}) that most matches are partial, 
covering typically less than half the query structure.
}
\end{footnotesize}
\end{figure}

% query = new.cas
% 
% Nterm = PIGTVIPIQHIRSVTGEPPRNPREIPIWLGRNAPAIDGVFPVTTPDLRCRIINAILGGNIGLSLTPGDCLTWDSAVATLFIRTHGT
% Cterm = FPMHQLGNVIKGIVDQEGVATAYTLGMMLSGQNYQLVSGIIRGYLPGQAVVTALQQRLDQEIDNQTRAETFIQHLNAVYEILGLNARGQSIRLE
% 
% Matches to PDB90
% 
%   No:  Chain   Z    rmsd lali nres  %id PDB  Description
%    1:  4x3x-A  5.0  3.1   66    82   11 PDB  MOLECULE: ACTIVITY-REGULATED CYTOSKELETON-ASSOCIATED PROTEI          
%    2|  3g29-A  3.7  2.7   60    77    8 PDB  MOLECULE: GAG POLYPROTEIN;                                           
%    3|  3g0v-A  3.7  2.9   62    76    8 PDB  MOLECULE: GAG POLYPROTEIN;                                           
%    4:  2v50-D  3.6  2.2   41   998    7 PDB  MOLECULE: MULTIDRUG RESISTANCE PROTEIN MEXB;                         
%    5:  3j39-i  3.6  2.5   40   113    3 PDB  MOLECULE: 60S RIBOSOMAL PROTEIN L10A-2;                              
%    6|  4ph2-A  3.6  3.2   69   127    7 PDB  MOLECULE: BLV CAPSID - N-TERMINAL DOMAIN;                            
%    7:  1iqp-E  3.6  3.8   69   326    7 PDB  MOLECULE: RFCS;                                                      
%    8:  4gco-A  3.6  3.7   55   120   11 PDB  MOLECULE: PROTEIN STI-1;                                             
%    9|  3g29-B  3.6  2.8   62    77    8 PDB  MOLECULE: GAG POLYPROTEIN;                                           
%   10|  3g1i-B  3.6  2.9   62    75    8 PDB  MOLECULE: GAG POLYPROTEIN;                                           
%   11|  3g21-A  3.6  2.8   60    77    8 PDB  MOLECULE: GAG POLYPROTEIN;                                           
%   12:  2a0u-A  3.5  3.1   68   374    4 PDB  MOLECULE: INITIATION FACTOR 2B;                                      
%   13:  1j7q-A  3.5  2.9   60    86    5 PDB  MOLECULE: CALCIUM VECTOR PROTEIN;                                    
%   14:  2a0u-B  3.5  8.1   80   367    4 PDB  MOLECULE: INITIATION FACTOR 2B;                                      
%   15:  1iqp-A  3.5  3.7   70   326    7 PDB  MOLECULE: RFCS;                                                      
%   16|  4ph0-C  3.5  4.6  101   199    8 PDB  MOLECULE: BLV CAPSID;                                                
%   17|  4ph0-D  3.5  4.2  101   198    8 PDB  MOLECULE: BLV CAPSID;                                                
%   18|  4ph2-B  3.5  3.3   69   127    7 PDB  MOLECULE: BLV CAPSID - N-TERMINAL DOMAIN;                            
%   19:  1sxj-B  3.4  3.5   65   316    3 PDB  MOLECULE: ACTIVATOR 1 95 KDA SUBUNIT;                                
%   20:  2afd-A  3.4  2.7   59    88   14 PDB  MOLECULE: PROTEIN ASL1650;                                           
%   21:  2yhe-A  3.4  2.8   40   639    8 PDB  MOLECULE: SEC-ALKYL SULFATASE;                                       
%   22:  4u8y-B  3.4  4.4   64  1033    3 PDB  MOLECULE: MULTIDRUG EFFLUX PUMP SUBUNIT ACRB;                        
%   23:  2yhe-C  3.4  7.1   70   634    6 PDB  MOLECULE: SEC-ALKYL SULFATASE;                                       
%   24:  2yhe-E  3.4  6.6   72   634    7 PDB  MOLECULE: SEC-ALKYL SULFATASE;                                       
%   25|  4coc-B  3.4  3.1   64    79    9 PDB  MOLECULE: CAPSID PROTEIN P24;                                        
%   26:  4av7-D  3.4  8.1   70   634    7 PDB  MOLECULE: SEC-ALKYLSULFATASE;                                        
%   27:  4mbq-F  3.4  3.2   42    50    5 PDB  MOLECULE: MOTILITY PROTEIN FIMV;                                     
%   28|  3g1g-B  3.4  2.9   61    75    8 PDB  MOLECULE: GAG POLYPROTEIN;                                           
%   29:  2w8a-A  3.3  2.0   43   531    7 PDB  MOLECULE: GLYCINE BETAINE TRANSPORTER BETP;                          
%   30|  3ce7-A  3.3  2.7   59    90    5 PDB  MOLECULE: SPECIFIC MITOCHODRIAL ACYL CARRIER PROTEIN;                
%   31:  1eia-A  3.3  2.8   62   207   11 PDB  MOLECULE: EIAV CAPSID PROTEIN P26;                                   
%   32:  1k6y-B  3.3  2.4   40   194    3 PDB  MOLECULE: INTEGRASE;                                                 
%   33:  2lo0-A  3.3  2.5   33    45    6 PDB  MOLECULE: UNCHARACTERIZED PROTEIN;                                   
%   34:  3w9j-B  3.3  4.3   64  1030    5 PDB  MOLECULE: MULTIDRUG RESISTANCE PROTEIN MEXB;                         
%   35:  4dx6-B  3.3  4.6   67  1033    4 PDB  MOLECULE: ACRIFLAVINE RESISTANCE PROTEIN B;                          
%   36:  1iqp-D  3.3  3.9   70   326    7 PDB  MOLECULE: RFCS;                                                      
%   37:  2afe-A  3.3  2.8   59    88   12 PDB  MOLECULE: PROTEIN ASL1650;                                           
%   38:  4av7-B  3.3  7.6   72   637    8 PDB  MOLECULE: SEC-ALKYLSULFATASE;                                        
%   39:  1iqp-B  3.3  4.1   70   326    7 PDB  MOLECULE: RFCS;                                                      
%   40:  1k6y-D  3.3  3.7   52   189    2 PDB  MOLECULE: INTEGRASE;                                                 
%   41:  1k6y-C  3.3  3.3   49   191    4 PDB  MOLECULE: INTEGRASE;                                                 
%   42:  1k6y-A  3.3  4.2   54   192    2 PDB  MOLECULE: INTEGRASE;                                                 
%   43:  2lo0-B  3.3  2.9   36    45    6 PDB  MOLECULE: UNCHARACTERIZED PROTEIN;                                   
%   44|  1afv-B  3.3  3.6   86   151   15 PDB  MOLECULE: HUMAN IMMUNODEFICIENCY VIRUS TYPE 1 CAPSID                 
%   45:  2o98-A  3.3 12.4   74   234   11 PDB  MOLECULE: 14-3-3-LIKE PROTEIN C;                                     
%   46|  4ph0-A  3.2  4.1  100   201    9 PDB  MOLECULE: BLV CAPSID;                                                
%   47|  1l6n-A  3.2  9.7   88   288   13 PDB  MOLECULE: GAG POLYPROTEIN;                                           
%   48:  3d5l-A  3.2  9.1   68   203    7 PDB  MOLECULE: REGULATORY PROTEIN RECX;                                   
%   49:  1mw7-A  3.2  2.3   41   220   10 PDB  MOLECULE: HYPOTHETICAL PROTEIN HP0162;                               
%   50:  2lni-A  3.2  3.5   43   133    2 PDB  MOLECULE: STRESS-INDUCED-PHOSPHOPROTEIN 1;                           
%   51:  2kc7-A  3.2  3.1   40    99   10 PDB  MOLECULE: BFR218_PROTEIN;                                            
%   52:  4dx5-B  3.2  4.6   65  1033    3 PDB  MOLECULE: ACRIFLAVINE RESISTANCE PROTEIN B;                          
%   53|  4coc-C  3.2  2.7   59    73    5 PDB  MOLECULE: CAPSID PROTEIN P24;                                        
%   54|  4coc-A  3.2  2.7   61    75    5 PDB  MOLECULE: CAPSID PROTEIN P24;                                        
%   55|  4cop-B  3.2  2.7   57    68    5 PDB  MOLECULE: CAPSID PROTEIN P24;                                        
%   56|  3lry-A  3.2  2.9   62    71    8 PDB  MOLECULE: HIV-1 CAPSID PROTEIN;                                      
%   57|  3lry-B  3.2  2.8   61    71    8 PDB  MOLECULE: HIV-1 CAPSID PROTEIN;                                      
%   58|  4ph0-B  3.2  4.2   88   188    9 PDB  MOLECULE: BLV CAPSID;                                                
%   59|  1afv-A  3.2  3.6   86   151   15 PDB  MOLECULE: HUMAN IMMUNODEFICIENCY VIRUS TYPE 1 CAPSID                 
%   60|  2gon-C  3.2  4.3   85   138   14 PDB  MOLECULE: CAPSID PROTEIN P24 (CA);                                   
%   61|  4ph3-B  3.2  3.1   64   115    8 PDB  MOLECULE: BLV CAPSID;                                                
%   62:  3ual-A  3.2 12.1   69   230    7 PDB  MOLECULE: 14-3-3 PROTEIN EPSILON;                                    
%   63:  1o9e-A  3.2 12.2   70   231   11 PDB  MOLECULE: 14-3-3-LIKE PROTEIN C;                                     
%   64:  1o9c-A  3.2 12.3   69   231   10 PDB  MOLECULE: 14-3-3-LIKE PROTEIN C;                                     
%   65:  3cu8-A  3.2 11.3   68   229    9 PDB  MOLECULE: 14-3-3 PROTEIN ZETA/DELTA;                                 
%   66:  3ubw-A  3.2 12.1   69   230    7 PDB  MOLECULE: 14-3-3 PROTEIN EPSILON;                                    
%   67:  1o9f-A  3.2 12.2   71   231    8 PDB  MOLECULE: 14-3-3-LIKE PROTEIN C;                                     
%   68:  4fj3-A  3.2 12.2   70   220    9 PDB  MOLECULE: 14-3-3 PROTEIN ZETA/DELTA;                                 
%   69|  3dph-B  3.2  2.9   63    80   10 PDB  MOLECULE: HIV-1 CAPSID PROTEIN;                                      
%   70|  3ds3-B  3.2  2.6   59    73   10 PDB  MOLECULE: HIV-1 CAPSID PROTEIN;                                      
%   71|  4ph1-A  3.2  3.2   63    73    6 PDB  MOLECULE: BLV CAPSID;                                                
%   72|  2y4z-A  3.1  3.8   68   135   10 PDB  MOLECULE: CAPSID PROTEIN P30;                                        
%   73|  5a9e-A  3.1  3.1   61   254    3 PDB  MOLECULE: DELTAMBD GAG PROTEIN;                                      
%   74:  2liu-A  3.1  2.9   58    99    7 PDB  MOLECULE: CURA;                                                      
%   75:  2o98-B  3.1  2.8   42   237   12 PDB  MOLECULE: 14-3-3-LIKE PROTEIN C;                                     
%   76:  3ph0-C  3.1  3.3   42    53   14 PDB  MOLECULE: ASCE;                                                      
%   77:  1s7e-A  3.1  2.0   40   147   10 PDB  MOLECULE: HEPATOCYTE NUCLEAR FACTOR 6;                               
%   78:  4dx7-B  3.1  4.5   65  1033    5 PDB  MOLECULE: ACRIFLAVINE RESISTANCE PROTEIN B;                          
%   79|  3ds2-B  3.1  3.2   70    84   11 PDB  MOLECULE: HIV-1 CAPSID PROTEIN;                                      
%   80|  2eia-B  3.1  2.9   61   204   11 PDB  MOLECULE: EIAV CAPSID PROTEIN P26;                                   
%   81|  4m0i-A  3.1  2.9   61    71    8 PDB  MOLECULE: HIV-1 CAPSID PROTEIN;                                      
%   82|  4u0d-L  3.1  3.9   83   191   14 PDB  MOLECULE: GAG POLYPROTEIN;                                           
%   83|  4ph3-A  3.1  3.1   65   115    8 PDB  MOLECULE: BLV CAPSID;                                                
%   84:  4wrq-A  3.1 12.3   71   220    8 PDB  MOLECULE: 14-3-3 PROTEIN ZETA/DELTA;                                 
%   85:  1o9d-A  3.1 11.4   68   230    9 PDB  MOLECULE: 14-3-3-LIKE PROTEIN C;                                     
%   86:  4n84-B  3.1 12.5   68   226    9 PDB  MOLECULE: 14-3-3 PROTEIN ZETA/DELTA;                                 
%   87:  4fl5-B  3.1 12.2   69   229    9 PDB  MOLECULE: 14-3-3 PROTEIN SIGMA;                                      
%   88|  1a43-A  3.1  2.9   61    72    8 PDB  MOLECULE: HIV-1 CAPSID;                                              
%   89:  2c63-D  3.1 10.9   63   233   10 PDB  MOLECULE: 14-3-3 PROTEIN ETA;                                        
%   90:  4n7g-A  3.1 12.4   69   229    7 PDB  MOLECULE: 14-3-3 PROTEIN ZETA/DELTA;                                 
%   91|  1baj-A  3.1  2.8   60    71    8 PDB  MOLECULE: GAG POLYPROTEIN;                                           
%   92:  3ph0-D  3.1  3.8   48    53   15 PDB  MOLECULE: ASCE;                                                      
%   93:  3uzd-A  3.1 12.2   69   229   13 PDB  MOLECULE: 14-3-3 PROTEIN GAMMA;                                      
%   94|  4ph1-C  3.1  3.1   64    79    6 PDB  MOLECULE: BLV CAPSID;                                                
%   95:  4n84-A  3.1 12.4   71   227    8 PDB  MOLECULE: 14-3-3 PROTEIN ZETA/DELTA;                                 
%   96:  3rmr-A  3.0  4.9   99   236   11 PDB  MOLECULE: AVIRULENCE PROTEIN;                                        
%   97:  2l0q-A  3.0  3.1   53    80    6 PDB  MOLECULE: ACYL CARRIER PROTEIN;                                      
%   98:  2br9-A  3.0  2.8   42   230   12 PDB  MOLECULE: 14-3-3 PROTEIN EPSILON;                                    
%   99:  2btp-A  3.0  2.8   42   248   14 PDB  MOLECULE: 14-3-3 PROTEIN TAU;                                        
%  100:  3axy-C  3.0  2.8   42   235   12 PDB  MOLECULE: PROTEIN HEADING DATE 3A;                                   
%     :
%  101|  1qrj-A  3.0  3.1   64   214    9 PDB  MOLECULE: HTLV-I CAPSID PROTEIN;                                     
%     :
%  104|  4xfx-A  3.0  4.0   82   216   15 PDB  MOLECULE: HIV-1 CAPSID PROTEIN;                                      
%  105|  4u0d-F  3.0  3.9   83   211   14 PDB  MOLECULE: GAG POLYPROTEIN;                                           
%  106|  2jpr-A  3.0  4.4   83   145   14 PDB  MOLECULE: GAG-POL POLYPROTEIN;                                       
%  107|  4xfz-A  3.0  3.1   79   213   15 PDB  MOLECULE: HIV-1 CAPSID PROTEIN;                                      
%  108|  2l6e-A  3.0  3.9   62    94    8 PDB  MOLECULE: CAPSID PROTEIN P24;                                        
%  109|  4xfy-A  3.0  3.9   83   216   14 PDB  MOLECULE: HIV-1 CAPSID PROTEIN;                                      
%  110|  4u0d-G  3.0  3.9   82   212   15 PDB  MOLECULE: GAG POLYPROTEIN;                                           
%  111|  4u0d-B  3.0  4.0   83   207   14 PDB  MOLECULE: GAG POLYPROTEIN;                                           
%     :
%  141|  4cop-A  2.9  2.9   64    83    8 PDB  MOLECULE: CAPSID PROTEIN P24;                                        
%  142|  2xv6-C  2.9  2.8   55    68    5 PDB  MOLECULE: CAPSID PROTEIN P24;                                        
%  143|  4u0c-A  2.9  4.0   83   210   14 PDB  MOLECULE: CAPSID PROTEIN P24;                                        
%  144|  4u0d-H  2.9  4.2   83   204   14 PDB  MOLECULE: GAG POLYPROTEIN;                                           
%  145|  2pwo-B  2.9  5.4   83   143   14 PDB  MOLECULE: GAG-POL POLYPROTEIN (PR160GAG-POL);                        
%  146|  2pwm-E  2.9  5.7   83   145   14 PDB  MOLECULE: GAG-POL POLYPROTEIN;                                       
%  147:  3upv-A  2.9  7.1   59   125    7 PDB  MOLECULE: HEAT SHOCK PROTEIN STI1;                                   
%  148|  4u0d-I  2.9  3.9   83   196   14 PDB  MOLECULE: GAG POLYPROTEIN;                                           
%  149|  4u0d-A  2.9  4.2   83   216   14 PDB  MOLECULE: GAG POLYPROTEIN;                                           
%  150|  3p05-E  2.9  3.2   78   199   14 PDB  MOLECULE: HIV-1 CA;                                                  
%     :
%  165|  3ds4-B  2.9  3.0   63    75   10 PDB  MOLECULE: HIV-1 CAPSID PROTEIN;                                      
%  166|  3ds4-A  2.9  2.8   59    77   10 PDB  MOLECULE: HIV-1 CAPSID PROTEIN;                                      
%     :
%  179|  3ds2-A  2.9  3.4   68    84   12 PDB  MOLECULE: HIV-1 CAPSID PROTEIN;                                      
%     :
%  183|  3ds3-A  2.9  2.7   58    73   10 PDB  MOLECULE: HIV-1 CAPSID PROTEIN;                                      
%     :
%  186|  2x82-A  2.8  3.3   67   145    9 PDB  MOLECULE: CAPSID PROTEIN P24;                                        
%     :
%  200|  3ds0-A  2.8  3.5   63    82   10 PDB  MOLECULE: HIV-1 CAPSID PROTEIN;                                      
%  201|  2xv6-A  2.8  2.9   60    75   10 PDB  MOLECULE: CAPSID PROTEIN P24;                                        
%     :
%  204|  3mge-A  2.8  4.0   81   204   15 PDB  MOLECULE: CAPSID PROTEIN P24;                                        
%  205|  1ak4-C  2.8  4.7   82   145   15 PDB  MOLECULE: CYCLOPHILIN A;                                             
%  206|  2pwm-H  2.8  5.4   81   145   15 PDB  MOLECULE: GAG-POL POLYPROTEIN;                                       
%  207|  4u0d-J  2.8  4.2   83   204   14 PDB  MOLECULE: GAG POLYPROTEIN;                                           
%     :
%  208|  1m9x-G  2.8  4.7   83   146   14 PDB  MOLECULE: CYCLOPHILIN A;                                             
%  209|  1e6j-P  2.8  3.1   78   209   15 PDB  MOLECULE: IMMUNOGLOBULIN;                                            
%  210|  4ph0-E  2.8  4.3   98   193    8 PDB  MOLECULE: BLV CAPSID;                                                
%     :
%  211|  4hkc-A  2.8 12.1   69   229    9 PDB  MOLECULE: 14-3-3 PROTEIN ZETA/DELTA;                                 
%  212|  3axy-J  2.8 12.1   71   234    8 PDB  MOLECULE: PROTEIN HEADING DATE 3A;                                   
%  213|  2c74-B  2.8 12.3   70   234   10 PDB  MOLECULE: 14-3-3 PROTEIN ETA;                                        
%  214|  4o46-C  2.8 12.2   69   235   12 PDB  MOLECULE: 14-3-3 PROTEIN GAMMA;                                      
%  215|  3axy-D  2.8 12.1   71   235    8 PDB  MOLECULE: PROTEIN HEADING DATE 3A;                                   
%  216|  3ds1-A  2.8  2.7   57    81    5 PDB  MOLECULE: HIV-1 CAPSID PROTEIN;                                      
%     :
%  241|  2pwm-D  2.7  5.4   82   145   13 PDB  MOLECULE: GAG-POL POLYPROTEIN;                                       
%  242|  4u0d-K  2.7  4.0   83   204   14 PDB  MOLECULE: GAG POLYPROTEIN;                                           
%     :
%  244|  2kod-A  2.7  3.6   65    88    5 PDB  MOLECULE: HIV-1 CA C-TERMINAL DOMAIN;                                
%     :
%  249|  2v4x-A  2.6  3.4   68   131    9 PDB  MOLECULE: CAPSID PROTEIN P27;                                        
%     :
%  272|  3j34-H  2.6  3.6   82   231   15 PDB  MOLECULE: CAPSID PROTEIN;                                            
%     :
%  274|  2jyl-A  2.6  6.8   64    84    9 PDB  MOLECULE: CAPSID PROTEIN P24 (CA);                                   
%     :
%  363|  3j34-d  2.4  3.9   83   231   13 PDB  MOLECULE: CAPSID PROTEIN;                                            
%     :
%  409|  3j34-6  2.3  4.4   84   231   14 PDB  MOLECULE: CAPSID PROTEIN;                                            
%  410|  3j34-Z  2.3  4.3   84   231   13 PDB  MOLECULE: CAPSID PROTEIN;                                            
%     :
%  471|  4e91-A  2.2  3.0   75   131   15 PDB  MOLECULE: GAG PROTEIN;                                               
%  472|  3j34-c  2.2  4.4   82   231   16 PDB  MOLECULE: CAPSID PROTEIN;                                            
%  473|  3j34-E  2.2  4.9   84   231   15 PDB  MOLECULE: CAPSID PROTEIN;                                            
%  474|  3j34-A  2.2  3.6   83   231   16 PDB  MOLECULE: CAPSID PROTEIN;                                            
%     :
%  548|  1u7k-A  2.1  4.9   79   131    8 PDB  MOLECULE: GAG POLYPROTEIN;                                           
%  549|  4ph0-F  2.1  4.5   97   199    8 PDB  MOLECULE: BLV CAPSID;                                                
%  550|  3j34-i  2.1  4.9   81   231   16 PDB  MOLECULE: CAPSID PROTEIN;                                            
%
 
\begin{figure}
\centering
\subfigure[{\tt 4x3x-A}]{
\epsfxsize=300pt \epsfbox{figs/full4x3x.super.eps}
}
\subfigure[{\tt 3g29-A}]{
\epsfxsize=300pt \epsfbox{figs/full3g29.super.eps}
}
\begin{footnotesize}
\caption{
\label{Fig:top2}
{\bf Top hits superposed}.
The top two \DALI\ hits to the full foamy virus capsid are shown as a \CA\ backbone (stereo pair) coloured using
the residue similarity score calculated by \SAP. (red = strong similarity, blue - none).
The amino terminus of the foamy structure is marked by a large ball and the other structure is distinguished
by small balls on its \CA\ atoms.
($a$) a cytoskeleton associated protein (fragment) of the arc/arg3.1 gene (PDB code:{\tt 4x3x-A}),
and 
($b$) the structure of the capsid C-terminal domain of the Rous scarcoma virus (PDB code:{\tt 3g29-A}). 
}
\end{footnotesize}
\end{figure}

The result of the DAIL search suggested that the Foamy virus structure shares some similarity with the
capsid structure of the ortho-viruses.  However, the matches consist only of a small number of
helices and appear barely more convincing than other matches to proteins that seem very unlikely
to have any meaningful connection to a viral capsid.   The preponderance of capsid matches
throughout the list of hits might seem to add some support to the relationship but may simply be 
a reflection of the number of capsid structures in the structure databank.

Adding confusion to the ortho/foamy relationship is the additional observation that 
the distribution of matches to the ortho-virus structures between the amino (N) and carboxy
(C) terminal domains are mixed.   For example; taking the top 10 matches, the N-terminal domain of the Foamy
structure aligns with 6 C-terminal domains and 4 N-terminal domains of the ortho virsuses
and the best match with the corresponding Foamy C-terminal domain aligns with an ortho N-terminal domain.

%
% Ortho virus: big N, small C
% Foamy smaller N, bigger C
% 
%                  :         :         :         :         |         :         :         :         :         100       :         :         :         :         |         :
% Full    PIGTVIPIQHIRSVTGEPPRNPREIPIWLGRNAPAIDGVFPVTTPDLRCRIINAILGGNIGLSLTPGDCLTWDSAVATLFIRTHGTFPMHQLGNVIKGIVDQEGVATAYTLGMMLSGQNYQLVSGIIRGYLPGQAVVTALQQRLDQEIDNQTRAETFI
% 3g29A-C ---------PWAD--IMQGPS--SFVDFANRLIKAVEGS---ARAPVIIDCFRQKSQPQQLI-------TTPGEIIKYVLDRQ---------------------------------------------------------------------------
% 3g0vA-C ---------PWAD--IMQGPS--SFVDFANRLIKAVEGSAL-ARAPVIIDCFRQKSQPQQLI-------TTPGEIIKYVLDRQ---------------------------------------------------------------------------
% 4ph2A-N RHRAW-ELQDIKK--EIEN----APGSVWIQTLRLAILQADP-TPADLEQLCQYIASPQTAHM-------YQNLWLQAWK-NLPT-------------------------------------------------------------------------
% 3g29B-C ---------PWAD--IMQGPS--SFVDFANRLIKAVEGSNL-ARAPVIIDCFRQKSQPQQLI-------TTPGEIIKYVLDRQ---------------------------------------------------------------------------
% 3g1iB-C ---------PWAD--IMQGPS--SFVDFANRLIKAVEGSDL-ARAPVIIDCFRQKSQPQQLI-------TTPGEIIKYVLDRQ---------------------------------------------------------------------------
% 3g21A-C ---------PWAD--IMQGPS--SFVDFANRLIKAVEGS---ARAPVIIDCFRQKSQPQQLI-------TTPGEIIKYVLDRQ---------------------------------------------------------------------------
% 4ph0C-N RHRAW-ELQDIKKEIENKA-----PGSQWIQTLRLAILQADP-TPADLEQLCQYIASPQTAHM-------YQNLWLQAWKNLPT----------------LQISLADNL----------PDGV--PKEPII-DSLS----------------------
% 4ph0D-N RHRAW-ELQDIKKE-IENK----APGSVWIQTLRLAILQADP-TPADLEQLCQYIASPQTAHM-------YQNLWLQAWKNLP-----------------QISL-ADNL----------PDGV--PKEPIISLSY-----------------------
% 4ph2B-N RHRAW-ELQDIKK--EIEN----APGSVWIQTLRLAILQADP-TPADLEQLCQYIASPQTAHM-------YQNLWLQAWK-NLPT-------------------------------------------------------------------------
% 4cocB-C ---------SILD--IRQGPK--PFRDYVDRFLKTLRAE---VKNWMTETLLVQNANPKTIL-LGPGA-----TLEEMMTACQGVG------------------------------------------------------------------------
% 3g1gB-C ---------PWAD--IMQGPS--SFVDFANRLIKAVEGSDL-ARAPVIIDCFRQKSQPQQLI-------TTPGEIIKYVLDR----------------------------------------------------------------------------
% 1afvB-N ----------------------------------------------------------------------IVQNL-------------SPRTLNAWVKVVEEK-VIPMFSALSE--GATPQDLNTMLNTVGGHQAAMQMLKETINEEASDIAYKRWIILG
% +
% 3nteA-C      YSPTSILD--IRQGPK EPFRDYVDRFYKTLRAEQASQEVKNWMTETLLVQNANPDCKTILKALGPAATLEEMMTACQGV
% 
% PIVQNIQGMVHQAISPRTLNAWVKVVEEKAFSPEVIPMFSALSEGATPQDLNTMLNTVGGHQAAMQMLKETINEEPRGSDIAGTTSTLQEQIGWMTNNPPIPVGEIYKRWIILGLNKIVRM
% ::        : :   : :    :  : :: :   :          :: ::           : :  :   :          :                                 `
% PIISEGNRNRHRAWALRELQDIKKEIENKAPSQVWIQTLRLAIADPTPADLEQLCQYIASQTAHMTSLTAAIAAAEAANTLQGFNPQNGTLTQQSAQPNAGDLRSQYQNLWLQAWKNLPTR
% 
% 
% 4PH2A-N PIISEGN-RNRHRAWALRELQDIKKEIENKAPGSQVWIQTLRLAILQADPTPADLEQLCQYIASPVDQTAHMTSLTAAIAAAEAANTLQGFNPQNGTLTQQSAQPNAGDLRSQYQNLWLQAWKNLPTR
% i                  |                                      || ||  
% 3nteA-N PIVQNIQGQMVHQAISPRTLNAWVKVVEEKAF-SPEVIP--MFSALSEGATPQDLNTMLNTVGGHQAAMQMLKETINEEAAEWDRVHPVHAGPIAPGQMREPRGSDIAGTTSTLQEQIGWMTNNPPIPVGEIYKRWIILGLNKIVRM
%      -C YSPTSILDIRQGPKEPFRDYVDRFYKTLRAEQASQEVKNWMTETLLVQNANPDCKTILKALGPAATLEEMMTACQGV
%
 
\begin{figure}
\centering
\subfigure[Full PDB]{
\label{Fig:daliFULL}
\rotatebox{270}{
\epsfxsize=140pt \epsfbox{figs/dali/daliFULL/DALIhitROC.ps}
}}
\subfigure[PDB-90]{
\label{Fig:daliNR90}
\rotatebox{270}{
\epsfxsize=140pt \epsfbox{figs/dali/daliNR90/DALIhitROC.ps}
}}
\begin{footnotesize}
\caption{
\label{Fig:rocs}
{\bf PDB capsid structure matches}.
The number of capsid structures identified by the \DALI\ program in ($a$) the full PDB and ($b$)
the 90\% non-redundant PDB (PDB-90) is shown as a ROC-like plot for queries using the full foamy capsid
structure (red), the carboxy terminal domain (green) and the amino terminal domain (blue). 
The number of capsid hits (Y-axis) is plotted against the order of all hits ranked by Z-score
down to a value of 2.   A curve approaching the top left corner indicates greater specificity 
and the extent of a curve to the right indicates the total number of hits.
}
\end{footnotesize}
\end{figure}

\subsubsection{Domain scans}

To clarify the domain match specificity, the two domains of the Foamy virus
(1--88 and 89--180, as defined in ref.\cite{TaylorWR99b}) were scanned separately using the \DALI\ program.  
The individual domains were much more specific at matching known capsid structures\footnote{
True/false hits were defined by protein descriptions with the words "CAPSID", "GAG" or "P24".
},   
both in the full PDB and PDB-90 collections as can be seen from the (un-normalised)
ROC-like plots in \Fig{rocs}.

The results of these scans strengthened the identification
of the relationship to the ortho capsids and supported the swapped specificity for the N-terminal
match of the Foamy structure with the C-terminal match of the ortho virus and {\em vica versa}, with all
top 12 hits of each domain matching their opposed counterpart.
The structure-based sequence alignments of each domain based on this equivalence are shown in \Fig{swap}.

\begin{figure}
\centering
\begin{singlespace}
\begin{tiny}
\begin{Verbatim}[frame=single]
Nter    PIGTVIPIQHIRSVTGEPPRNPREIPIWLGRNAPAIDGVFPVTTPDLRCRIINAILGGNIGLSLTPGDCLTWDSAVATLFIRTHGTFP
3g1gA   ---------PWAD--IMQGPS--SFVDFANRLIKAVEGSDL-ARAPVIIDCFRQKSQPQQLI--PSTL-TTPGEIIKYVLDRQK----
3tirA   ---------PWAD--IMQGPS--SFVDFANRLIKAVEGSDL-ARAPVIIDCFRQKSQPQQLI-------TTPGEIIKYVLDRQ-----
3g1iA   ---------PWAD--IMQGPS--SFVDFANRLIKAVEGSDL-ARAPVIIDCFRQKSQPQQLI----TLTT-PGEIIKYVLDRQ-----
3g29A   ---------PWAD--IMQGPS--SFVDFANRLIKAVEGS---ARAPVIIDCFRQKSQPQQLI-------TTPGEIIKYVLDRQ-----
3g0vA   ---------PWAD--IMQGPS--SFVDFANRLIKAVEGSAL-ARAPVIIDCFRQKSQPQQLI-------TTPGEIIKYVLDRQ-----
3g29B   ---------PWAD--IMQGPS--SFVDFANRLIKAVEGSNL-ARAPVIIDCFRQKSQPQQLI-------TTPGEIIKYVLDRQ-----
3g1iB   ---------PWAD--IMQGPS--SFVDFANRLIKAVEGSDL-ARAPVIIDCFRQKSQPQQLI-------TTPGEIIKYVLDRQ-----
3g26A   ---------PWAD--IMQGPS--SFVDFANRLIKAVEGS---CRAPVIIDCFRQKSQPQQLI-------TTPGEIIKYVLDRQ-----
3dtjC   ---------SILD--IRQGPK--EPFRDYVDRFYKTLR--VKNW--MTATLLVQNANPD-TILKGPGA--TLEEMMTA-CQGV-----
3dtjB   ---------SILD--IRQGPK--EPFRDYVDRFYKTLR--VKNW--MTATLLVQNANPD-TILKGPGA--TLEEMMTA-CQGV-----
3dtjA   ---------SILD--IRQGPK--EPFRDYVDRFYKTLR--VKNW--MTATLLVQNANPD-TILKGPGA--TLEEMMTA-CQGV-----
3g21A   ---------PWAD--IMQGPS--SFVDFANRLIKAVEGSDL-ARAPVIIDCFRQKSQPQQLI-------TTPGEIIKYVLDRQ-----

Cter    MHQLGNVIKGIVDQEGVATAYTLGMMLSGQNYQLVSGIIRGYLPGQAVVTALQQRLDQEIDNQTRAETFIQHLNAVYEILGLNARGQSIRL
1l6nA   SPRTLNAWVKVVEEKA-IPMFSALSE---GATPDLNTMLNTVGGHQAAMQMLKETINEEA--EIYKRWIILGLNKIVRMYS------PTSI
3j34U   SPRTLNAWVKVVEEKA-IPMFSALSE--GATPQDLNTMLNTVGGHQAAMQMLKETINEEA--EIYKRWIILGLNKIVRMY-------SPTS
4u0bF   SPRTLNAWVKVVEEKA-IPMFSALSC--GATPQDLNTMLNTVGGHQAAMQMLKETINEEA--EIYKRWIILGLNKIVRMY-------SPTS
4u0bG   SPRTLNAWVKVVEEK--IPMFSALSC--GATPQDLNTMLNTVGGHQAAMQMLKETINEEA--EIYKRWIILGLNKIVRMY-------SPTS
3h4eB   SPRTLNAWVKVVEEK--IPMFSALSC--GATPQDLNTMLNTVGGHQAAMQMLKETINEEA--EIYKRWIILGLNKIVRMY-------SPTS
2jprA   SPRTLNAWVKVVEEKA-IPMFSALSE--GATPQDLNTMLNTVGGHQAAMQMLKETINEEA--EIYKRWIILGLNKIVRMY-----------
1afvB   SPRTLNAWVKVVEEKAVIPMFSALSE--GATPQDLNTMLNTVGGHQAAMQMLKETINEEA--EIYKRWIILGLNKIVRMY-------SPTS
4u0bE   SPRTLNAWVKVVEEK--IPMFSALSC--GATPQDLNTMLNTV-GHQAAMQMLKETINEEA--EIYKRWIILGLNKIVRMY-------SPTS
4u0bK   SPRTLNAWVKVVEEK--IPMFSALSC--GATPQDLNTMLNTVGGHQAAMQMLKETINEEA--EIYKRWIILGLNKIVRMY-------SPTS
4u0bH   SPRTLNAWVKVVEEK--IPMFSALSC--GATPQDLNTMLNTVGGHQAAMQMLKETINEEA--EIYKRWIILGLNKIVRMY-------SPTS
2gonA   SPRTLNAWVKVVEEK-VIPXFSALSE--GATPQDLNTXLNTVGGHQAAXQXLKETINEEA--EIYKRWIILGLNKIVRXYS----------
1afvA   SPRTLNAWVKVVEEKAVIPMFSALSE--GATPQDLNTMLNTVGGHQAAMQMLKETINEEA--EIYKRWIILGLNKIVRMY-------SPTS
\end{Verbatim}
\end{tiny}
\end{singlespace}
\begin{footnotesize}
\caption{
\label{Fig:swap}
{\bf Top domain similarity alignments}.
The sequence alignments are shown for the top 12 capsid domain matches found by the \DALI\ program 
using the foamy virus capsid N and C domains separately as a query over the full PDB.
The sequence of the N-terminal domain ({\tt N-ter}) is shown at the top of the first alignment block and the
sequences of the C-terminal domain ({\tt C-ter}) at the top of the second block.   The sequences of the
ortho-virsuses aligned below these all come from the "swapped" relationship of C and N terminal domains,
respectively.  (Note: the alignment is determined by structure, not sequence similarity.)
}
\end{footnotesize}
\end{figure}

Although domain transposition is not impossible in viral genomes,  it is sufficiently
unexpected to warrant deeper investigation, especially as it is hard to imagine how an ancestral
capsid protein could tolerate such a large rearrangement and still pack to form a competent shell.

%
% >3nte
%    Nter PIVQNIQGQMVHQAI
%         SPRTLNAWVKVVEEKAFSPEVIPMFSALSEGATPQDLNTMLNTVGGHQAAMQMLKETINEEAAEWDRVHPVHAGPIAPGQMREPRGSDIAGTTSTLQEQIGWMTNNPPIPVGEIYKRWIILGLNKIVRM
%    Cter YSPTSILDIRQGPKEPFRDYVDRFYKTLRAEQASQEVKNWMTETLLVQNANPDCKTILKALGPAATLEEMMTACQGV
% wtaylor@wt:~/ianpdbs/dali$ cut -c 6-175 newN.aln | head -14
%                  :         :         :         :         |         :         :         :        
% Nter    PIGTVIPIQHIRSVTGEPPRNPREIPIWLGRNAPAIDGVFPVTTPDLRCRIINAILGGNIGLSLTPGDCLTWDSAVATLFIRTHGTFP
% 3g1gA   ---------PWAD--IMQGPS--SFVDFANRLIKAVEGSDL-ARAPVIIDCFRQKSQPQQLI--PSTL-TTPGEIIKYVLDRQK----
% 3tirA   ---------PWAD--IMQGPS--SFVDFANRLIKAVEGSDL-ARAPVIIDCFRQKSQPQQLI-------TTPGEIIKYVLDRQ-----
% 3g1iA   ---------PWAD--IMQGPS--SFVDFANRLIKAVEGSDL-ARAPVIIDCFRQKSQPQQLI----TLTT-PGEIIKYVLDRQ-----
% 3g29A   ---------PWAD--IMQGPS--SFVDFANRLIKAVEGS---ARAPVIIDCFRQKSQPQQLI-------TTPGEIIKYVLDRQ-----
% 3g0vA   ---------PWAD--IMQGPS--SFVDFANRLIKAVEGSAL-ARAPVIIDCFRQKSQPQQLI-------TTPGEIIKYVLDRQ-----
% 3g29B   ---------PWAD--IMQGPS--SFVDFANRLIKAVEGSNL-ARAPVIIDCFRQKSQPQQLI-------TTPGEIIKYVLDRQ-----
% 3g1iB   ---------PWAD--IMQGPS--SFVDFANRLIKAVEGSDL-ARAPVIIDCFRQKSQPQQLI-------TTPGEIIKYVLDRQ-----
% 3g26A   ---------PWAD--IMQGPS--SFVDFANRLIKAVEGS---CRAPVIIDCFRQKSQPQQLI-------TTPGEIIKYVLDRQ-----
% 3dtjC   ---------SILD--IRQGPK--EPFRDYVDRFYKTLR--VKNW--MTATLLVQNANPD-TILKGPGA--TLEEMMTA-CQGV-----
% 3dtjB   ---------SILD--IRQGPK--EPFRDYVDRFYKTLR--VKNW--MTATLLVQNANPD-TILKGPGA--TLEEMMTA-CQGV-----
% 3dtjA   ---------SILD--IRQGPK--EPFRDYVDRFYKTLR--VKNW--MTATLLVQNANPD-TILKGPGA--TLEEMMTA-CQGV-----
% 3g21A   ---------PWAD--IMQGPS--SFVDFANRLIKAVEGSDL-ARAPVIIDCFRQKSQPQQLI-------TTPGEIIKYVLDRQ-----
% +
% 3nteA-C      YSPTSILD--IRQGPK--EPFRDYVDRFYKTLRaeqasqeVKNWMTETLLVQNANPDckTILKALGPAaTLEEMMTACQGV
% 
%      -N PIVQNIQGQMVHQAISPRTLNAWVKVVEEKAFSPEVIPMFSALSEGATPQDLNTMLNTVGGHQAAMQMLKETINEEAAEWDRVHPVHAGPIAPGQMREPRGSDIAGTTSTLQEQIGWMTNNPPIPVGEIYKRWIILGLNKIVRM
% 3nteA-C YSPTSILDIRQGPKEPFRDYVDRFYKTLRAEQASQEVKNWMTETLLVQNANPDCKTILKALGPAATLEEMMTACQGV
% 
% wtaylor@wt:~/ianpdbs/dali$ cut -c 6-175 newC.aln | head -14
%                  :         :         :         :         |         :         :         :         :  
% Cter    MHQLGNVIKGIVDQEGVATAYTLGMMLSGQNYQLVSGIIRGYLPGQAVVTALQQRLDQEIDNQTRAETFIQHLNAVYEILGLNARGQSIRL
% 1l6nA   SPRTLNAWVKVVEEKA-IPMFSALSE---GATPDLNTMLNTVGGHQAAMQMLKETINEEA--EIYKRWIILGLNKIVRMYS------PTSI
% 3j34U   SPRTLNAWVKVVEEKA-IPMFSALSE--GATPQDLNTMLNTVGGHQAAMQMLKETINEEA--EIYKRWIILGLNKIVRMY-------SPTS
% 4u0bF   SPRTLNAWVKVVEEKA-IPMFSALSC--GATPQDLNTMLNTVGGHQAAMQMLKETINEEA--EIYKRWIILGLNKIVRMY-------SPTS
% 4u0bG   SPRTLNAWVKVVEEK--IPMFSALSC--GATPQDLNTMLNTVGGHQAAMQMLKETINEEA--EIYKRWIILGLNKIVRMY-------SPTS
% 3h4eB   SPRTLNAWVKVVEEK--IPMFSALSC--GATPQDLNTMLNTVGGHQAAMQMLKETINEEA--EIYKRWIILGLNKIVRMY-------SPTS
% 2jprA   SPRTLNAWVKVVEEKA-IPMFSALSE--GATPQDLNTMLNTVGGHQAAMQMLKETINEEA--EIYKRWIILGLNKIVRMY-----------
% 1afvB   SPRTLNAWVKVVEEKAVIPMFSALSE--GATPQDLNTMLNTVGGHQAAMQMLKETINEEA--EIYKRWIILGLNKIVRMY-------SPTS
% 4u0bE   SPRTLNAWVKVVEEK--IPMFSALSC--GATPQDLNTMLNTV-GHQAAMQMLKETINEEA--EIYKRWIILGLNKIVRMY-------SPTS
% 4u0bK   SPRTLNAWVKVVEEK--IPMFSALSC--GATPQDLNTMLNTVGGHQAAMQMLKETINEEA--EIYKRWIILGLNKIVRMY-------SPTS
% 4u0bH   SPRTLNAWVKVVEEK--IPMFSALSC--GATPQDLNTMLNTVGGHQAAMQMLKETINEEA--EIYKRWIILGLNKIVRMY-------SPTS
% 2gonA   SPRTLNAWVKVVEEK-VIPXFSALSE--GATPQDLNTXLNTVGGHQAAXQXLKETINEEA--EIYKRWIILGLNKIVRXYS----------
% 1afvA   SPRTLNAWVKVVEEKAVIPMFSALSE--GATPQDLNTMLNTVGGHQAAMQMLKETINEEA--EIYKRWIILGLNKIVRMY-------SPTS
% +
% 3nteA-N SPRTLNAWVKVVEEKAFSPEVIPMFSALSEGATPQDLNTMLNTVGGHQAAMQMLKETINEEAAEWDRVHPVHAGPIAPGQMREPRGSDIAGTTSTLQEQIGWMTNNPPIPVGEIYKRWIILGLNKIVRM
% 
% 3nteA-N PIVQNIQGQMVHQAI
%         SPRTLNAWVKVVEEKAFSPEVIPMFSALSEGATPQDLNTMLNTVGGHQAAMQMLKETINEEAAEWDRVHPVHAGPIAPGQMREPRGSDIAGTTSTLQEQIGWMTNNPPIPVGEIYKRWIILGLNKIVRM
%      -C YSPTSILDIRQGPKEPFRDYVDRFYKTLRAEQASQEVKNWMTETLLVQNANPDCKTILKALGPAATLEEMMTACQGV
% 

\subsection{Structural alignment significance}

\subsubsection{Reversed-structure searches}

For each comparison, the DALI program calculates a Z-score, combining a estimation of significance
with protein length normalisation.   The program reports all matches over Z=2, however, when the
proteins are small and especially when the structures being compared are both predominantly
alpha-helical in nature, then matches over this cutoff include many functionally unrelated
hits where the similarity has arisen through the fortuitous alignment of a few helices.
(For example; the top hit when scanning with the C-terminal domain is a non-capsid structure).

To calculate a stricter cutoff on score, we created a decoy probe by reversing the
alpha-carbon backbone then reconstructing the full atomic structure, using a simple algorithm
to regenerate a full backbone\footnote{
Note that reversing the \CA\ backbone does not change the chirality of the \A-helices
but as \DALI\ requires a full atomic backbone, this must be restored on the reversed chain.
}).
\Fig{revs} plots the ranked Z-scores for the separate (native) foamy domains.
% N red=T, cyan=F: C magenta=T, green=F
As would be expected, the larger C-terminal domain has hits with a higher significance than the
smaller N-terminal domain:  the former covers the range Z=2.5 to Z=5 over the true hits (magenta
dots) wheras the latter tracks a slmilar profile running one Z-value unit lower (2--4 over true
red dots).  Plotting the Z-scores against the log of their rank produces almost linear traces
for the hits from the PDB-90, making it easy to compare N-domain (red/cyan dots) with C-domain 
(magenta/green dots) (for T/F hits) in \Fig{revs}.

The equivalent scans with the reversed domain structures, using both the foamy and ortho (HIV) structures 
(neither of which should have any particular relationship to the capsid or any other natural protein)
also found hits with high Z-scores (black and blue points in \Fig{revs}, respectively).
When compared with the native domains (\Fig{revs}), these decoys had a profile that tracked mostly above
the N-terminal native domain but below the C-terminal domain.  However, with the latter domain, this
was only distinct in the hits to the full PDB whereas with the PDB-90, the native domain was only clearly 
better over the top 10 matches, half of which were to non-capsid structures.

% set style line  1 lt rgb "black"
% set style line  2 lt rgb "red" 
% set style line  3 lt rgb "cyan"
% set style line  4 lt rgb "magenta"
% set style line  5 lt rgb "green"
% set style line  6 lt rgb "blue"
% plot [-0.1:][1.9:5.1]\
% '3etnN.dat' u (log($1)):(($3-0.04))    ls 6  pt 7 ps 0.7 not, \
% '3etnC.dat' u (log($1)):(($3-0.04))    ls 6  pt 5 ps 0.7 not, \
% 'wenN.dat' u (log($1)):(($3-0.02))     ls 1  pt 7 ps 0.7 not, \
% 'wenC.dat' u (log($1)):(($3-0.02))     ls 1  pt 5 ps 0.7 not, \
% 'capFnewN.dat' u (log($2)):(($4))      ls 3  pt 7 ps 0.7 not, \
% 'capTnewN.dat' u (log($2)):(($4+0.02)) ls 2  pt 7 ps 0.7 not, \
% 'capFnewC.dat' u (log($2)):(($4))      ls 5  pt 7 ps 0.7 not, \
% 'capTnewC.dat' u (log($2)):(($4+0.02)) ls 4  pt 7 ps 0.7 not
% set terminal postscript color
% set output 'rawDALIdom.ps'
 
\begin{figure}
\centering
\subfigure[full PDB]{
\label{Fig:daliNR90}
\rotatebox{270}{
\epsfxsize=140pt \epsfbox{figs/dali/daliFULL/rawDALIdom.ps}
}}
\subfigure[PDB-90]{
\label{Fig:daliFULL}
\rotatebox{270}{
\epsfxsize=140pt \epsfbox{figs/dali/daliNR90/rawDALIdom.ps}
}}
\begin{footnotesize}
\caption{
\label{Fig:revs}
{\bf Ranked \DALI\ scores with decoys}.
The \DALI\ scores (Y-axis) are plotted against the log$_{10}$ of their ranked position in the 
list of hits (X-axis) with the amino-terminal domain (N) as T=red, F=cyan dots and the carboxy-terminal domain (C)
as T=magenta and F=green dots, where T is a true capsid hit and F is a false hit to a non-capsid protein.
Four sets of decoys are compared to these, consisting of the reversed foamy capsid domains in
black and the reversed HIV capsid domains in dark-blue (with a circle = N and a square = C domains in both).
The \DALI\ score for each set of hits has been slightly displaced to prevent coincident dots from being
obscured.  (This happens because of the integral number of residues and the \DALI\ score being specified
to only one decimal place).
Note: the top scoring hit in both plots is the {\tt 4x3x-A} protein (Figure 2$a$) which is correctly coloured cyan (not green).
}
\end{footnotesize}
\end{figure}

The results with the simple reversed decoy using \DALI\, suggested that the match of the foamy virus domains to the
ortho virus capsid N-terminal domain may be due to chance and that the match to the C-terminal domain looks 
meaningful if based on the hits to the full PDB but may be only marginal based on the PDB-90 hits. 

However, both the N and C terminal domains pocess a degree of internal symmetry which gives
rise to a partial match with their reversed 'doppleganger' decoys.   The N-terminal domain superposed on its decoy
had an RMSD of 5.4/60 (\AA/\CA s) and 5.5/24 for the C-terminal domain.   The higher symmetry of the smaller
domain may be sufficient to explain its poor level of specificity seen in \Fig{revs} and to try and resolve this
ambiguity, a more diverse set of decoys were generated based on cyclic permutation and segment swapping combined
with chain reversal \cite{TaylorWT97e}.

\begin{figure}
\centering
\subfigure[N]{
\epsfxsize=150pt \epsfbox{figs/twoN.eps}
}
\subfigure[C]{
\epsfxsize=150pt \epsfbox{figs/twoC.eps}
}
\begin{footnotesize}
\caption{
\label{Fig:tows}
{\bf Native/decoy similarity}.
When superposed using the program \SAP, both N-terminal (left) and C-terminal (right) domains
have some degree of similarity to their reversed decoy 'doppleganger', which is more marked
for the N domain.   The superposed structures are coloured by the \SAP\ residue-level score as
red = high similarity, blue = low.  The N domain has roughly 60 equivalent \CA\ positions
compared to only 24 in the larger C domain.
}
\end{footnotesize}
\end{figure}

\subsubsection{Customised decoy comparisons}

To improve the statistical analysis of the foamy/ortho capsid similarity, we employed a method
based on the generation of a population of customised 'decoy' models to provide a background distribution
of unrelated protein scores \cite{Taylor}.  This method retains the advantage of the simple
reversed structures where every comparison that constitutes the random pool is between two models
of the same size and secondary structure composition as the pair of native structures being compared.
For this study we collected 10 capsid N-terminal domains and 7 C-terminal domains, each of which 
were compared with the foamy N-terminal domain and the foamy C-terminal domain.
(The structures are identified in Table 1 with full details in the Methods section).

For each domain pair to be compared, decoys were created using cyclic permutation
and segment swapping with chain reversal to generate a family of customised decoys for each
comparison \cite{Taylor}.  All pairs of forward/reversed decoys were then compared with each pair being
drawn from the pool based on the two native structures.  This ensures that the native domains
(which may have different lengths) are always evaluated against a decoy pair with the same
length combination.   (See Methods section for details).   All the decoy comparisons, of which
there are typically 150--300 for each comparison,  can then be compared to the native pair on a
plot of RMSD againsed the number of matched residues (\CA\ atoms).   An example is shown in
\Fig{sapit} for the comparison of the HIV1 structure \cite{ian} domains against the foamy domains.
 
\begin{figure}
\centering
\subfigure[orthoN+foamyN]{
\label{Fig:sapitNN}
\rotatebox{270}{
\epsfxsize=140pt \epsfbox{figs/sapitNN.ps}
}}
\subfigure[orthoN+foamyC]{
\label{Fig:sapitNC}
\rotatebox{270}{
\epsfxsize=140pt \epsfbox{figs/sapitNC.ps}
}}
\subfigure[orthoC+foamyN]{
\label{Fig:sapitCN}
\rotatebox{270}{
\epsfxsize=140pt \epsfbox{figs/sapitCN.ps}
}}
\subfigure[orthoC+foamyC]{
\label{Fig:sapitCC}
\rotatebox{270}{
\epsfxsize=140pt \epsfbox{figs/sapitCC.ps}
}}
\begin{footnotesize}
\caption{
\label{Fig:sapit}
{\bf ortho/foamy domains compared with customised decoys}.
Each amino (N), carboxy (C) domain combination between the ortho virus capsid structure (HIV1)
and the foamy capsid structure is plotted as a line for increasingly large subsets of matched
positions against their RMSD (Y-axis), as in Figure 2.  The point on this line marks the lowest
$a$-value (\Eqn{fit}), however, to be consistent with the decoy data, the full alignment length
was used.  The decoy comparison data (blue) is plotted in a variety of symbols
with each representing a different combination of decoy construction.  The dashed blue lines
(which are the same in all plots) mark the approximate 10$^{th}$ percentile boundaries of
the decoy generated distributions,  with {\tt a} = 1.7 (upper) and  {\tt a} = 0.8 (lower). 
(See Methods section for details).
}
\end{footnotesize}
\end{figure}

\begin{figure}
\centering
\subfigure[]{
\label{Fig:normHIV}
\rotatebox{270}{
\epsfxsize=140pt \epsfbox{figs/fitsCN/plothiv.ps}
}}
\subfigure[]{
\label{Fig:normALL}
\rotatebox{270}{
\epsfxsize=140pt \epsfbox{figs/fitsCN/plotall.ps}
}}
\begin{footnotesize}
\caption{
\label{Fig:norms}
{\bf ortho-C and foamy-N domain comparison statistics}.
The $a$-value (normalised RMSD) for the comparison of the ortho-C and foamy-N decoy domains
(Figure 8($b$)) are plotted as a frequency distribution (red) along with a bell-shaped
Normal distribution curve (green) with matching mean ($\mu$) and spread ($\sigma$).
Part ($a$) shows the distribution for the HIV1 C-terminal domain ($\mu = 1.23$) and spread ($\sigma = 0.17$)
with the position of the native structure comparison plotted as a blue (inverted) triangle.  
Its position lies 0.64 units below the mean giving a Z-score of $0.64/0.17 = 3.76$.
Part ($b$) shows the combined data from seven representative viruses (in Table 1).
These data comprise two distributions, that of the combined decoys and also the much smaller
distribution of native scores (blue triangles).   This allows a T-test to be made on the
significance of their separation.
}
\end{footnotesize}
\end{figure}

%
%  Run sapit.csh for each domain against the foamy N and C
%
%wtaylor@wt:~/ianpdbs/sapit$ cat rundom.csh
%# 1 = domain
%
%mkdir $argv[1]
%cd $argv[1]
%ln -s ~/util
%ln -s ~/sapit main
%ln -s .. home
%ln -s ../pdb
%tcsh main/sapit.csh $argv[1] foamN > $argv[1]N.log
%tcsh main/sapit.csh $argv[1] foamC > $argv[1]C.log
%
% and do this for each capsid domain
%
%wtaylor@wt:~/ianpdbs/sapit$ cat rundoms.csh
%foreach dom (`cat [NC]term.list`)
%	echo $dom
%	tcsh rundom.csh $dom
%end
%
% convert the comparisons into Z-scores using sapit/fits1.c
% fits1 reads the true and the random scores and discards
% random pairs outside +/-10 of the true alignment length.
% It then converts the RMS:Length (r:x) values into... 
% a[i] = r/(sqrt(x)*(1-exp(-x*x/damp)));
% The mean and std. of a[] give the Z-score for the true.
%
% NB the values of margin = 10 and damp = 30 were used
%
%wtaylor@wt:~/ianpdbs/sapit$ more scoreZ.csh
%rm score*.dat
%set foamN = `echo foamN`
%set foamC = `echo foamC`
%foreach dom (`cat Nterm.list`)
%	echo $dom
%	set score = `main/fits1 $dom/$dom$foamN` 
%	echo $dom+N $score >> scoreNN.dat
%	set score = `main/fits1 $dom/$dom$foamC` 
%	echo $dom+C $score >> scoreNC.dat
%end
%foreach dom (`cat Cterm.list`)
%	echo $dom
%	set score = `main/fits1 $dom/$dom$foamN` 
%	echo $dom+N $score >> scoreCN.dat
%	set score = `main/fits1 $dom/$dom$foamC` 
%	echo $dom+C $score >> scoreCC.dat
%end
%

\subsubsection{Statistical analysis of the decoy comparisons}

The quality of the comparisons in \Fig{sapit} can be quantified as a combination of their
RMSD ($R$) and the number of matched (superposed) positions ($N$).   However, as explained in 
the Methods section, it is easier for statistical analysis to combine this pair of numbers
as a single number,  called the $a$-value (\Eqn{fit}), which is the scaling factor that
causes a theoretical curve to pass through the point ($R,N$).

When expressed by a single $a$-value
all the data points in a comparison, such as \Fig{sapitCN},
can be plotted as a frequency histogram and examined to see if they approximate a Normal
distribution.   The distributions were found to be a good fit to unskewed Gaussians and
so were treated as normal distributions (rather than extreme value distributions
that have also been considered previously as a model for random structure comparison scores
\cite{Levitt,Taylor}).    The frequency data from the comparison of the orthoN domain from
HIV1 and the foamyC domain (\Fig{sapitCN}) is shown in \Fig{normHIV} along with a Normal  
distribution that has the same mean ($\mu$) and standard deviation ($\sigma$) as the data.
On this plot, the value of $a$ (\Eqn{fit}) for the comparison of the
native pair of domains is also plotted (blue triangle) and from its position, a Z-score
can be calculated.

In this way, the significance of all combinations of the native ortho and foamy domain
superpositions were calculated, using the background distribution of 'customised' decoy
comparisons based on each individual native pair.   The resulting Z-scores ($\sigma$ units)
are collected in \Tab{Zscores}.   The degree of similarity between the domains ranged from
less than 1$\sigma$ to over 5$\sigma$, with the latter (highly significant) result being
obtained for both a swapped (NC) and forward (CC) combination.   However, of the top 12
scores, only three now came from swapped pairings. 

\begin{table}
\centering
\begin{tabular}{r|lll|lll|}
$a$  & \multicolumn{6}{c|}{\bf ortho-N} \\
     & \multicolumn{3}{c|}{\bf foamy-N} & \multicolumn{3}{c|}{\bf foamy-C}  \\
%::::::::::::::
%scoreNN.dat
%::::::::::::::
%BLV_N+N damp = 30.000000 margin = 10 n = 251 mean = 1.315207 stdv = 0.170234 y = 0.550254 mean-y = 0.764954 z = 4.493551
%HIV1_N+N damp = 30.000000 margin = 10 n = 312 mean = 1.387546 stdv = 0.219740 y = 0.573903 mean-y = 0.813643 z = 3.702746
%HML2_N+N damp = 30.000000 margin = 10 n = 264 mean = 1.265025 stdv = 0.225198 y = 0.777153 mean-y = 0.487873 z = 2.166418
%HTLV_N+N damp = 30.000000 margin = 10 n = 400 mean = 1.324345 stdv = 0.181467 y = 0.592956 mean-y = 0.731389 z = 4.030429
%JSRV_N+N damp = 30.000000 margin = 10 n = 225 mean = 1.243907 stdv = 0.201215 y = 1.063649 mean-y = 0.180259 z = 0.895854
%MLV_N+N damp = 30.000000 margin = 10 n = 326 mean = 1.312253 stdv = 0.184224 y = 0.751373 mean-y = 0.560880 z = 3.044558
%MPMV_N+N damp = 30.000000 margin = 10 n = 269 mean = 1.304454 stdv = 0.189464 y = 0.565209 mean-y = 0.739246 z = 3.901779
%PSIV_N+N damp = 30.000000 margin = 10 n = 285 mean = 1.341958 stdv = 0.193299 y = 0.620753 mean-y = 0.721205 z = 3.731022
%RELIK_N+N damp = 30.000000 margin = 10 n = 234 mean = 1.377677 stdv = 0.200315 y = 0.639005 mean-y = 0.738672 z = 3.687560
%RSV_N+N damp = 30.000000 margin = 10 n = 204 mean = 1.299841 stdv = 0.242518 y = 0.542528 mean-y = 0.757312 z = 3.122707
\hline \hline
virus  & pool & $a$-value & Z-score & pool & $a$-value & Z-score \\
\hline
BLV    &  251  & 0.550 & {\bf 4.49} &  184  & 0.400 &      3.67  \\
HIV1   &  312  & 0.573 &      3.70  &  213  & 0.402 &      3.69  \\
HML2   &  264  & 0.777 &      2.17  &  196  & 0.438 & {\bf 4.59} \\
HTLV   &  400  & 0.592 & {\bf 4.03} &  328  & 0.457 & {\bf 4.01} \\
JSRV   &  225  & 1.063 &      0.90  &  190  & 0.601 &      3.24  \\
MLV    &  326  & 0.751 &      3.04  &  188  & 0.508 &      3.15  \\
MPMV   &  269  & 0.565 & {\bf 3.90} &  185  & 0.523 &      2.92  \\
PSIV   &  285  & 0.621 &      3.73  &  235  & 0.369 & {\bf 5.02} \\
RELIK  &  234  & 0.639 &      3.69  &  237  & 0.700 &      3.29  \\
RSV    &  204  & 0.543 &      3.12  &  239  & 0.526 &      3.54  \\
\hline \hline
%::::::::::::::
%scoreNC.dat
%::::::::::::::
%BLV_N+C damp = 30.000000 margin = 10 n = 184 mean = 1.234132 stdv = 0.227439 y = 0.399576 mean-y = 0.834556 z = 3.669370
%HIV1_N+C damp = 30.000000 margin = 10 n = 213 mean = 1.304811 stdv = 0.244515 y = 0.402160 mean-y = 0.902651 z = 3.691602
%HML2_N+C damp = 30.000000 margin = 10 n = 196 mean = 1.337005 stdv = 0.195668 y = 0.438086 mean-y = 0.898919 z = 4.594095
%HTLV_N+C damp = 30.000000 margin = 10 n = 328 mean = 1.283603 stdv = 0.205965 y = 0.457005 mean-y = 0.826598 z = 4.013291
%JSRV_N+C damp = 30.000000 margin = 10 n = 190 mean = 1.284324 stdv = 0.210871 y = 0.601744 mean-y = 0.682580 z = 3.236956
%MLV_N+C damp = 30.000000 margin = 10 n = 188 mean = 1.240526 stdv = 0.232555 y = 0.507666 mean-y = 0.732859 z = 3.151332
%MPMV_N+C damp = 30.000000 margin = 10 n = 185 mean = 1.204718 stdv = 0.233608 y = 0.522996 mean-y = 0.681723 z = 2.918237
%PSIV_N+C damp = 30.000000 margin = 10 n = 235 mean = 1.346684 stdv = 0.194710 y = 0.369476 mean-y = 0.977208 z = 5.018795
%RELIK_N+C damp = 30.000000 margin = 10 n = 237 mean = 1.358031 stdv = 0.199553 y = 0.700138 mean-y = 0.657893 z = 3.296824
%RSV_N+C damp = 30.000000 margin = 10 n = 239 mean = 1.285199 stdv = 0.214446 y = 0.525590 mean-y = 0.759609 z = 3.542199
\vspace{10pt}
\end{tabular}
\begin{tabular}{r|lll|lll|}
$b$  & \multicolumn{6}{c|}{\bf ortho-C} \\
     & \multicolumn{3}{c|}{\bf foamy-N} & \multicolumn{3}{c|}{\bf foamy-C}  \\
%::::::::::::::
%scoreCN.dat
%::::::::::::::
%BLV6_C+N damp = 30.000000 margin = 10 n = 144 mean = 1.269846 stdv = 0.167814 y = 0.763161 mean-y = 0.506686 z = 3.019336
%BLV_C+N damp = 30.000000 margin = 10 n = 154 mean = 1.268382 stdv = 0.202504 y = 0.579920 mean-y = 0.688462 z = 3.399738
%HIV1_C+N damp = 30.000000 margin = 10 n = 157 mean = 1.229855 stdv = 0.169134 y = 0.593804 mean-y = 0.636051 z = 3.760632
%HIV6_C+N damp = 30.000000 margin = 10 n = 179 mean = 1.314433 stdv = 0.168393 y = 0.779728 mean-y = 0.534705 z = 3.175344
%HML2_C+N damp = 30.000000 margin = 10 n = 185 mean = 1.269326 stdv = 0.177262 y = 0.732750 mean-y = 0.536576 z = 3.027011
%HTLV_C+N damp = 30.000000 margin = 10 n = 156 mean = 1.267269 stdv = 0.151450 y = 0.684654 mean-y = 0.582614 z = 3.846900
%RSV_C+N damp = 30.000000 margin = 10 n = 155 mean = 1.226569 stdv = 0.207357 y = 0.448186 mean-y = 0.778383 z = 3.753835
\hline \hline
virus  & pool & $a$-value & Z-score & pool & $a$-value & Z-score \\
\hline
BLV6   &  144  & 0.763 &      3.02  &  212  & 0.709 & {\bf 4.05} \\
BLV    &  154  & 0.578 &      3.40  &  204  & 0.556 & {\bf 4.05} \\
HIV1   &  157  & 0.593 &      3.76  &  174  & 0.705 &      3.36  \\
HIV6   &  179  & 0.780 &      3.18  &  177  & 0.640 & {\bf 4.38} \\
HML2   &  185  & 0.732 &      3.02  &  184  & 0.676 & {\bf 3.90} \\
HTLV   &  156  & 0.685 & {\bf 3.85} &  163  & 0.694 &      2.81  \\
RSV    &  155  & 0.448 &      3.75  &  235  & 0.403 & {\bf 5.01} \\
\hline \hline
%::::::::::::::
%scoreCC.dat
%::::::::::::::
%BLV6_C+C damp = 30.000000 margin = 10 n = 212 mean = 1.345119 stdv = 0.157218 y = 0.708869 mean-y = 0.636250 z = 4.046926
%BLV_C+C damp = 30.000000 margin = 10 n = 204 mean = 1.305845 stdv = 0.185196 y = 0.556341 mean-y = 0.749504 z = 4.047075
%HIV1_C+C damp = 30.000000 margin = 10 n = 174 mean = 1.291238 stdv = 0.174192 y = 0.705527 mean-y = 0.585711 z = 3.362449
%HIV6_C+C damp = 30.000000 margin = 10 n = 177 mean = 1.419853 stdv = 0.178084 y = 0.639918 mean-y = 0.779935 z = 4.379582
%HML2_C+C damp = 30.000000 margin = 10 n = 184 mean = 1.291423 stdv = 0.157803 y = 0.676068 mean-y = 0.615355 z = 3.899511
%HTLV_C+C damp = 30.000000 margin = 10 n = 163 mean = 1.259457 stdv = 0.201546 y = 0.693736 mean-y = 0.565720 z = 2.806904
%RSV_C+C damp = 30.000000 margin = 10 n = 235 mean = 1.300217 stdv = 0.179102 y = 0.403158 mean-y = 0.897059 z = 5.008660
\end{tabular}
\begin{footnotesize}
\caption{
\label{Tab:Zscores}
{\bf ortho and foamy domain comparison Z-score statistics}.
For each amino (N) and carboxy (C) domain pair between an ortho virus structure and the foamy virus capsid structure,
a {\bf Z-score} is calculated based on the {\bf a-value} (\Eqn{fit}) derived from the comparison RMSD and length,
relative to the {\bf pool} of background decoy comparisons.   The ortho {\bf virus} identity is indicated by the 
code to the left, full details of which can be found in the Methods section.
The top 12 Z-scores are high-lighted in bold.
}
\end{footnotesize}
\end{table}

To obtain a more quantitative consensus
for the amino/amino (NN) and carboxy/carboxy (CC) versus the swapped domain pairings
(NC and CN), the raw results were combined for each pairing, giving now not just a
single value compared to a distribution but two distrubutions. (\Fig{normALL}).   For these data,
a significance was calculated using Student's T-test, the values of which are given
in (\Tab{Ttest}).

%wtaylor@wt:~/ianpdbs/sapit$ tcsh scoreZ.csh 1 15
%BLV_N
%HIV1_N
%HML2_N
%HTLV_N
%JSRV_N
%MLV_N
%MPMV_N
%PSIV_N
%RELIK_N
%RSV_N
%BLV6_C
%BLV_C
%HIV1_C
%HIV6_C
%HML2_C
%HTLV_C
%RSV_C
%
%stdev.= 0.18877
%Zscore= 3.30819
%T-tests
%
%N+foamN
% in1 = 10 maln = 72 naln = 80 mean1 = 0.666767 stdv1 = 0.160854 in2 = 833 mean2 = 1.316649 stdv2 = 0.211645
%Avg: 6.67e-01 < 1.32e+00 Tprob=4.62e-21 **
%StD: 1.61e-01 = 2.12e-01 Fprob=1.84e-01 
%N+foamC
% in1 = 10 maln = 79 naln = 87 mean1 = 0.492081 stdv1 = 0.101925 in2 = 996 mean2 = 1.290236 stdv2 = 0.220791
%Avg: 4.92e-01 < 1.29e+00 Tprob=4.09e-10 **
%StD: 1.02e-01 < 2.21e-01 Fprob=7.37e-03 **
%C+foamN
% in1 = 7 maln = 59 naln = 74 mean1 = 0.650558 stdv1 = 0.117477 in2 = 992 mean2 = 1.247814 stdv2 = 0.189033
%Avg: 6.51e-01 < 1.25e+00 Tprob=2.35e-16 **
%StD: 1.17e-01 = 1.89e-01 Fprob=1.12e-01 
%C+foamC
% in1 = 7 maln = 60 naln = 80 mean1 = 0.622420 stdv1 = 0.111796 in2 = 1189 mean2 = 1.300281 stdv2 = 0.176998
%Avg: 6.22e-01 < 1.30e+00 Tprob=3.81e-23 **
%StD: 1.12e-01 = 1.77e-01 Fprob=1.20e-01 
%

\begin{table}
\centering
\begin{tabular}{c|c|c|}
             &          {\bf orthoN}           &          {\bf orthoC}           \\
\hline \hline
             & {\tt Avg: 6.67e-01 < 1.32e+00 } & {\tt Avg: 6.51e-01 < 1.25e+00 } \\
             & {\tt Tprob = 4.62e-21 **      } & {\tt Tprob = 2.35e-16 **      } \\
{\bf foamyN} &                                 &                                 \\
             & {\tt StD: 1.61e-01 = 2.12e-01 } & {\tt StD: 1.17e-01 = 1.89e-01 } \\
             & {\tt Fprob = 1.84e-01         } & {\tt Fprob = 1.12e-01         } \\
\hline
             & {\tt Avg: 4.92e-01 < 1.29e+00 } & {\tt Avg: 6.22e-01 < 1.30e+00 } \\
             & {\tt Tprob = 4.09e-10 **      } & {\tt Tprob = 3.81e-23 **      } \\
{\bf foamyC} &                                 &                                 \\
             & {\tt StD: 1.02e-01 < 2.21e-01 } & {\tt StD: 1.12e-01 = 1.77e-01 } \\
             & {\tt Fprob = 7.37e-03 **      } & {\tt Fprob = 1.20e-01         } \\
\hline \hline
\end{tabular}
\begin{footnotesize}
\caption{
\label{Tab:Ttest}
{\bf ortho and foamy capsid domain comparison T-test significance}.
For each combinfation of domains between the ortho and foamy viruses, the probability
is given that the two means from each distribution ({\tt Avg} values) were sampled
from the same distribution.  (i.e., that the native and decoy comparisons are
not distinct).   All domain pairings are extreemly significant.   An F-test was used to
test if the standard deviations ({\tt Std}) of each sample were distinct and if not,
the a T-test was made on the assumption of equal standard deviations.
}
\end{footnotesize}
\end{table}

From these results, it can be seen that all the four possible pairings are
highly significant with probabilities ranging from $10^{-10}$ to over $10^{-20}$.
It is also clear that the two swapped pairings (NC and CN) have lower probabilities
than the forward pairings (NN and CC).   Combining the probabilities ($P$) as:
$\Delta P = \log{_10}(P_{NN} P_{CC}) - log_{10}(P_{NC} P_{CN})$,
gives a value of 18 (42.7 - 25.0 = 17.7) which means that the swapped pairing is almost 18
orders of magnitude (or a billion,billion times) less likely than the forward pairing.
The unexpected swapped pairing, which was indicated originally by the \DALI\ results, now seems
less likely.  The preferred, and biologically more reasonable, result is that the ortho virus
domain are related to the foamy virus domains as a result of genetic divergence from
a common, double domain anscestor. 

% combining
%
%wtaylor@wt:~/ianpdbs/sapit$ echo 4.62e-21 3.83e-23 | awk '{print $1,$2, $1*$2, -log($1*$2)}'
%4.62e-21 3.83e-23 1.76946e-43 98.4405
%wtaylor@wt:~/ianpdbs/sapit$ echo 4.09e-10 2.35e-16 | awk '{print $1,$2, $1*$2, -log($1*$2)}'
%4.09e-10 2.35e-16 9.6115e-26 57.6043
%
%wtaylor@wt:~/ianpdbs/sapit$ echo 4.62e-21 3.83e-23 | awk '{print $1,$2, $1*$2, -log($1*$2)/log(10)}'
%4.62e-21 3.83e-23 1.76946e-43 42.7522
%wtaylor@wt:~/ianpdbs/sapit$ echo 4.09e-10 2.35e-16 | awk '{print $1,$2, $1*$2, -log($1*$2)/log(10)}'
%4.09e-10 2.35e-16 9.6115e-26 25.0172
%
% 42.7 - 25.0 = 17.7, 10**18 = 10**9 * 10**9 = billion,billion 
%

\subsection{Internal duplication}

Despite a simpler N/N, C/C equivalance, the swapped pairing of N/C and C/N (ortho/foamy)
domains, nevertheless, retains a high structural significance and this
suggests that the two domains are derived from a common ancestral structure, probably
as the result of a prior gene-duplication event that has been retained more clearly
in the less embellished foamy virus structures.   Comparing the two foamy domins gives
a Z-score of 2.077 sigma which, although of marginal significance, supports this model.
(\Fig{fitsNC}($a,b$)).

Such a relationship between the foamy domains implies an equivalent relationship
in the ortho viruses and a similar comparison in structures of their N and C domains
finds matches with Z-scores ranging from 2 to 4.   As with the comparison of the 
ortho and foamy structures, these can be pooled to allow a joint T-test to be applied.  
This gave a probability of $10^{-8}$ that the true N/C domain
comparisons were drawn from the decoy distribution, adding strong support to the
hypothesis of an ancient gene duplication occuring before the split of the ortho 
and foamy virus families. (\Fig{fitsNC}($c,d$), blue triangles).

This test was applied only to the comparison of domains between viruses with 
known structures for both domains, however, it is not unreasonable to compare
amino and carboxy domains across all viruses.  The longer loops in the ortho virus
domains gives greater scope of structural variation and a wide range of variation
was seen ranging from RMSD values under 4 to over 12.  
When normalised for length ($a$-value from \Eqn{fit}) and partial matches under
60 positions excluded, a distinct cluster remains between $a=0.5\ldots0.8$ (4...6\AA\ 
RMSD) but still with a long tail to higher values.
Despite this tail, the T-test on the distributions is highly significant at $2.7xi \times 10^{-17}$.

One of the better N/C ortho similarities is shown in \Fig{final}($a$), along with the
N/C ortho domain superposition in \Fig{final}($b$).
%
% foamyN vs foamyC
%
%wtaylor@wt:~/ianpdbs/sapit$ main/fits1 foamN+foamC 30 10
% damp = 30.000000 margin = 10
% n = 157 mean = 1.247504 stdv = 0.164467 y = 0.905853 mean-y = 0.341651 z = 2.077320
%
%wtaylor@wt:~/ianpdbs/sapit$ main/fits1 HIV1_N+HIV1_C 30 10
% damp = 30.000000 margin = 10
% n = 317 mean = 1.348796 stdv = 0.237078 y = 0.879874 mean-y = 0.468922 z = 1.977924
%
%
%wtaylor@wt:~/ianpdbs/sapit$ main/fits1 BLV_N+BLV_C 30 10
% damp = 30.000000 margin = 10
% n = 249 mean = 1.314609 stdv = 0.189414 y = 0.548639 mean-y = 0.765970 z = 4.043883
%wtaylor@wt:~/ianpdbs/sapit$ main/fits1 HIV1_N+HIV1_C 30 10
% damp = 30.000000 margin = 10
% n = 317 mean = 1.348796 stdv = 0.237078 y = 0.879874 mean-y = 0.468922 z = 1.977924
%wtaylor@wt:~/ianpdbs/sapit$ main/fits1 HML2_N+HML2_C 30 10
% damp = 30.000000 margin = 10
% n = 191 mean = 1.334377 stdv = 0.203411 y = 0.559553 mean-y = 0.774824 z = 3.809163
%wtaylor@wt:~/ianpdbs/sapit$ main/fits1 HTLV_N+HTLV_C 30 10
% damp = 30.000000 margin = 10
% n = 223 mean = 1.254856 stdv = 0.212137 y = 0.797973 mean-y = 0.456883 z = 2.153713
%wtaylor@wt:~/ianpdbs/sapit$ main/fits1 RSV_N+RSV_C 30 10
% damp = 30.000000 margin = 10
% n = 85 mean = 1.251280 stdv = 0.329498 y = 0.681746 mean-y = 0.569534 z = 1.728493
%wtaylor@wt:~/ianpdbs/sapit$ main/stest orthoNC 30 10
% argv[1] = orthoNC damp = 30.000000
% in1 = 6 maln = 49 naln = 78 mean1 = 0.728939 stdv1 = 0.156379 in2 = 1864 mean2 = 1.281632 stdv2 = 0.239511
%Avg: 7.29e-01 < 1.28e+00 Tprob=1.88e-08 **
%StD: 1.56e-01 = 2.40e-01 Fprob=1.69e-01 
%
 
\begin{figure}
\centering
\subfigure[foamy N+C]{
\label{Fig:foamyNCsapit}
\rotatebox{270}{
\epsfxsize=140pt \epsfbox{figs/foamyNC/sapit.ps}
}}
\subfigure[foamy fit]{
\label{Fig:foamyNCfits}
\rotatebox{270}{
\epsfxsize=140pt \epsfbox{figs/foamyNC/fits.ps}
}}
\subfigure[ortho N+C]{
\label{Fig:orthoNCsapit}
\rotatebox{270}{
\epsfxsize=140pt \epsfbox{figs/orthoNCall/sapit.ps}
}}
\subfigure[ortho fit]{
\label{Fig:orthoNCfits}
\rotatebox{270}{
\epsfxsize=140pt \epsfbox{figs/orthoNCall/fits.ps}
}}
\begin{footnotesize}
\caption{
\label{Fig:fitsNC}
{\bf N and C domains compared with customised decoys}.
$a$) The N and C domains of the foamy virus (black) compared to decoys (blue) with ($b$) the derived frequency plot with the
native comparison marked by a blue triangle.  (See legend to \Fig{sapit} for details).
$c$) The N and C domains of the ortho virus combinations (black) with ($d$) the derived frequency plot showing the
native comparison for pairs from the same virus (blue triangles) with the distribution of all native pairs
shown as a scattered frequency plot (blue line).
(See Methods section for details).
}
\end{footnotesize}
\end{figure}

\begin{figure}
\centering
\subfigure[ortho]{
\epsfxsize=300pt \epsfbox{figs/orthoNCall/ortho.eps}
}
\subfigure[foamy]{
\epsfxsize=300pt \epsfbox{figs/orthoNCall/foamy.eps}
}
\begin{footnotesize}
\caption{
\label{Fig:final}
{\bf Amino and carboxy domains superposed}.
$a$ ortho virus domains and
$b$ ortho virus domains are shown as a stereo pair with
their \CA\ backbones coloured by
the residue similarity score calculated by \SAP. (red = strong similarity, blue - none).
The amino terminal domain is distinguished by small balls on its \CA\ positions and
the amino terminus lies to the top in both panels.
}
\end{footnotesize}
\end{figure}

\subsection{Fold-space representation}

To summarise the structural relationships amontg the ortho and foamy domains, the matrix
of pairwise comparisons was projected into a three-dimensional fold-space.  (See methods
for details).   This produces a best ivisual representation of the RMSD values between domains.

As can be seen from \Fig{space}, the N and C domains of the ortho viruses form distinct
clusters with the foamy C domain lying closer to the ortho C-domain cluster.   The faomy
N-domain, however, maintains a fairly equal distance from both ortho domain clusters but
lies closer to its C-terminal partner.

\begin{figure}
\centering
\epsfxsize=300pt \epsfbox{figs/foldspace/space.eps}
\begin{footnotesize}
\caption{
\label{Fig:final}
{\bf Fold-space representation of all domains}.
All the viral domains considered in the paper were projected into a 3D fold-space representing
the relationship of their \SAP\ weighted RMSD values.   The domains are coloured as:
foamyN = cyan, foamyC = red, orthoN = green and ortho C = magenta. 
}
\end{footnotesize}
\end{figure}
