\section{Conclusions}

\subsection{Structure comparison}

\subsubsection{Pairwise significance}

The comparison of small domains that are largely composed of \AHs\ presents a challenging problem in how
to interpret the significance of the RMSD values.  As the individual helical secondary structure elements
(SSEs) constitute a sizeable fraction of the domain, it takes only the chance alignment of a few helices
to result in a low RMSD over a large proportion of the structure, giving an apparently meaingful result.  

The use of
the customised decoy sets attempts to avoided this problem by recreating a large number of possible folds
that were generated using the same (reconnected) SSEs.   Moreover, to avoid any chance recreation of native fragments,
each comparison always involved the comparison of a native (forward) chain direction with a reversed chain.
Using these models, a background distribution of decoy/decoy comparisons allowed us to calculate Z-scores
for each native/native comparison with the advantage that every comparison in the background distribution
involved two models with the same length, density and secondary composition as the native pair.
These values indicated a clearly significant relationship between the foamy and ortho structures.   

\subsubsection{Direct or transposed domain order?}

However, the Z-scores did not point to a clear resolution of whether the domains should have a direct
correspondance (NN and CC match) or a transposed relationship (NC and CN) with significant individual
matches found across all pairings.  Testing for a bias towards more significant like-domain pairings (NN,CC)
in the list of similarities ranked by Z-score confirmed the visual bias towards a direct correspondance
but only at a marginal level of significance (around 0.05).  By contrast, the application of a T-test on the
combined raw comparison data returned a very clear distinction between the direct and the transposed
relationships, clearly favouring the more natural forward order.  

Although the `astronomic' probabilities calculated by the T-test seem very convincing, they
must be viewed in the light of the much lower probabilities calculated from the asymmetry
statistics.   Both calculations involve assumptions and are limited by the small number of known
structures so neither can be taken as definitive.    It would seem likely that the ``true'' level of 
significance may lie somewhere between the two results but as both point in the direction of the NN and CC 
domain order, there is no reason to adopt the more unexpected transposed domain order. 

\subsection{Evolutionary implications}

On the basis of these structural comparisons, and a variety of functional assays described elsewhere, we can
conclude that the central domain of the spumavirus Gag gene encodes a polypeptide sequence related to that of
the corresponding region of orthoretroviruses, CA. It therefore seems reasonable to suppose that the last
common ancestor of orthoretroviruses and spumaviruses possessed such a sequence. Moreover this region appears
to be made up from two related subdomains suggesting a gene duplication event in a common precursor.

In our initial search of the foamy virus capsid using the DALI program, we made
the curious observation that the strongest similarity of the foamy virus capsid
was with the ARC protein (Activity-Regulated Cytoskeleton-associated protein) that
is active in neural synaptic growth and activity (and several other developmental associated functions).
The ARC protein has widespread and clear (non-gag) sequence homologues as far back as insects and probably
deeper, giving it a very ancient origin somewhere close to the metazoan root \cite{CampillosMet06}.
If ARC is considered to be a relic of an ancient Ty3/Gypsy retrotransposon \cite{ZhangWet15}, preserved
as a 'living fossil' in the genomes of metazoa, this relationship would suggest an equally
ancient origin for the foamy virus. 

Alternatively,
the foamy viruses may have co-opted an ARC protein to facilitate budding and their escape from the cell.
As it is believed that the Ty3/Gypsy family of intracellular retrotransposons gave rise to retroviruses
\cite{LlorensCet08}, 
it will therefore be of considerable interest to determine whether such elements possess CA proteins with
a two-domain structure. Finally, it is worth noting that the Gag protein of Ty3 is significantly shorter 
that that of the retroviruses and it is possible that the N-terminal domains of the orthoretroviruses 
and spumaviruses were co-opted at different times to facilitate budding from the cell surface. 
If so, the very different structures of this region in the orthoretroviruses and spumaviruses might suggest 
independent acquisition events. 
