\section{Introduction}
Foamy virus capsids are of interest.

\section{Results}

\subsection{DALI searches}

\subsubsection{Full chain scan}

A scan of the full-length foamy structure using the DALI server
{\tt http://ekhidna.biocenter.helsinki.fi/dali_server}
over the 90\% non-redundant protein structure databank
identified a wide selection of retroviral capsid structures.  In the ranked list of structue hits,
capsids identified from position 2 to position 550.

% 
% Nterm = PIGTVIPIQHIRSVTGEPPRNPREIPIWLGRNAPAIDGVFPVTTPDLRCRIINAILGGNIGLSLTPGDCLTWDSAVATLFIRTHGT
% Cterm = FPMHQLGNVIKGIVDQEGVATAYTLGMMLSGQNYQLVSGIIRGYLPGQAVVTALQQRLDQEIDNQTRAETFIQHLNAVYEILGLNARGQSIRLE
% 
% Matches to PDB90
% 
%   No:  Chain   Z    rmsd lali nres  %id PDB  Description
%    1:  4x3x-A  5.0  3.1   66    82   11 PDB  MOLECULE: ACTIVITY-REGULATED CYTOSKELETON-ASSOCIATED PROTEI          
%    2|  3g29-A  3.7  2.7   60    77    8 PDB  MOLECULE: GAG POLYPROTEIN;                                           
%    3|  3g0v-A  3.7  2.9   62    76    8 PDB  MOLECULE: GAG POLYPROTEIN;                                           
%    4:  2v50-D  3.6  2.2   41   998    7 PDB  MOLECULE: MULTIDRUG RESISTANCE PROTEIN MEXB;                         
%    5:  3j39-i  3.6  2.5   40   113    3 PDB  MOLECULE: 60S RIBOSOMAL PROTEIN L10A-2;                              
%    6|  4ph2-A  3.6  3.2   69   127    7 PDB  MOLECULE: BLV CAPSID - N-TERMINAL DOMAIN;                            
%    7:  1iqp-E  3.6  3.8   69   326    7 PDB  MOLECULE: RFCS;                                                      
%    8:  4gco-A  3.6  3.7   55   120   11 PDB  MOLECULE: PROTEIN STI-1;                                             
%    9|  3g29-B  3.6  2.8   62    77    8 PDB  MOLECULE: GAG POLYPROTEIN;                                           
%   10|  3g1i-B  3.6  2.9   62    75    8 PDB  MOLECULE: GAG POLYPROTEIN;                                           
%   11|  3g21-A  3.6  2.8   60    77    8 PDB  MOLECULE: GAG POLYPROTEIN;                                           
%   12:  2a0u-A  3.5  3.1   68   374    4 PDB  MOLECULE: INITIATION FACTOR 2B;                                      
%   13:  1j7q-A  3.5  2.9   60    86    5 PDB  MOLECULE: CALCIUM VECTOR PROTEIN;                                    
%   14:  2a0u-B  3.5  8.1   80   367    4 PDB  MOLECULE: INITIATION FACTOR 2B;                                      
%   15:  1iqp-A  3.5  3.7   70   326    7 PDB  MOLECULE: RFCS;                                                      
%   16|  4ph0-C  3.5  4.6  101   199    8 PDB  MOLECULE: BLV CAPSID;                                                
%   17|  4ph0-D  3.5  4.2  101   198    8 PDB  MOLECULE: BLV CAPSID;                                                
%   18|  4ph2-B  3.5  3.3   69   127    7 PDB  MOLECULE: BLV CAPSID - N-TERMINAL DOMAIN;                            
%   19:  1sxj-B  3.4  3.5   65   316    3 PDB  MOLECULE: ACTIVATOR 1 95 KDA SUBUNIT;                                
%   20:  2afd-A  3.4  2.7   59    88   14 PDB  MOLECULE: PROTEIN ASL1650;                                           
%   21:  2yhe-A  3.4  2.8   40   639    8 PDB  MOLECULE: SEC-ALKYL SULFATASE;                                       
%   22:  4u8y-B  3.4  4.4   64  1033    3 PDB  MOLECULE: MULTIDRUG EFFLUX PUMP SUBUNIT ACRB;                        
%   23:  2yhe-C  3.4  7.1   70   634    6 PDB  MOLECULE: SEC-ALKYL SULFATASE;                                       
%   24:  2yhe-E  3.4  6.6   72   634    7 PDB  MOLECULE: SEC-ALKYL SULFATASE;                                       
%   25|  4coc-B  3.4  3.1   64    79    9 PDB  MOLECULE: CAPSID PROTEIN P24;                                        
%   26:  4av7-D  3.4  8.1   70   634    7 PDB  MOLECULE: SEC-ALKYLSULFATASE;                                        
%   27:  4mbq-F  3.4  3.2   42    50    5 PDB  MOLECULE: MOTILITY PROTEIN FIMV;                                     
%   28|  3g1g-B  3.4  2.9   61    75    8 PDB  MOLECULE: GAG POLYPROTEIN;                                           
%   29:  2w8a-A  3.3  2.0   43   531    7 PDB  MOLECULE: GLYCINE BETAINE TRANSPORTER BETP;                          
%   30|  3ce7-A  3.3  2.7   59    90    5 PDB  MOLECULE: SPECIFIC MITOCHODRIAL ACYL CARRIER PROTEIN;                
%   31:  1eia-A  3.3  2.8   62   207   11 PDB  MOLECULE: EIAV CAPSID PROTEIN P26;                                   
%   32:  1k6y-B  3.3  2.4   40   194    3 PDB  MOLECULE: INTEGRASE;                                                 
%   33:  2lo0-A  3.3  2.5   33    45    6 PDB  MOLECULE: UNCHARACTERIZED PROTEIN;                                   
%   34:  3w9j-B  3.3  4.3   64  1030    5 PDB  MOLECULE: MULTIDRUG RESISTANCE PROTEIN MEXB;                         
%   35:  4dx6-B  3.3  4.6   67  1033    4 PDB  MOLECULE: ACRIFLAVINE RESISTANCE PROTEIN B;                          
%   36:  1iqp-D  3.3  3.9   70   326    7 PDB  MOLECULE: RFCS;                                                      
%   37:  2afe-A  3.3  2.8   59    88   12 PDB  MOLECULE: PROTEIN ASL1650;                                           
%   38:  4av7-B  3.3  7.6   72   637    8 PDB  MOLECULE: SEC-ALKYLSULFATASE;                                        
%   39:  1iqp-B  3.3  4.1   70   326    7 PDB  MOLECULE: RFCS;                                                      
%   40:  1k6y-D  3.3  3.7   52   189    2 PDB  MOLECULE: INTEGRASE;                                                 
%   41:  1k6y-C  3.3  3.3   49   191    4 PDB  MOLECULE: INTEGRASE;                                                 
%   42:  1k6y-A  3.3  4.2   54   192    2 PDB  MOLECULE: INTEGRASE;                                                 
%   43:  2lo0-B  3.3  2.9   36    45    6 PDB  MOLECULE: UNCHARACTERIZED PROTEIN;                                   
%   44|  1afv-B  3.3  3.6   86   151   15 PDB  MOLECULE: HUMAN IMMUNODEFICIENCY VIRUS TYPE 1 CAPSID                 
%   45:  2o98-A  3.3 12.4   74   234   11 PDB  MOLECULE: 14-3-3-LIKE PROTEIN C;                                     
%   46|  4ph0-A  3.2  4.1  100   201    9 PDB  MOLECULE: BLV CAPSID;                                                
%   47|  1l6n-A  3.2  9.7   88   288   13 PDB  MOLECULE: GAG POLYPROTEIN;                                           
%   48:  3d5l-A  3.2  9.1   68   203    7 PDB  MOLECULE: REGULATORY PROTEIN RECX;                                   
%   49:  1mw7-A  3.2  2.3   41   220   10 PDB  MOLECULE: HYPOTHETICAL PROTEIN HP0162;                               
%   50:  2lni-A  3.2  3.5   43   133    2 PDB  MOLECULE: STRESS-INDUCED-PHOSPHOPROTEIN 1;                           
%   51:  2kc7-A  3.2  3.1   40    99   10 PDB  MOLECULE: BFR218_PROTEIN;                                            
%   52:  4dx5-B  3.2  4.6   65  1033    3 PDB  MOLECULE: ACRIFLAVINE RESISTANCE PROTEIN B;                          
%   53|  4coc-C  3.2  2.7   59    73    5 PDB  MOLECULE: CAPSID PROTEIN P24;                                        
%   54|  4coc-A  3.2  2.7   61    75    5 PDB  MOLECULE: CAPSID PROTEIN P24;                                        
%   55|  4cop-B  3.2  2.7   57    68    5 PDB  MOLECULE: CAPSID PROTEIN P24;                                        
%   56|  3lry-A  3.2  2.9   62    71    8 PDB  MOLECULE: HIV-1 CAPSID PROTEIN;                                      
%   57|  3lry-B  3.2  2.8   61    71    8 PDB  MOLECULE: HIV-1 CAPSID PROTEIN;                                      
%   58|  4ph0-B  3.2  4.2   88   188    9 PDB  MOLECULE: BLV CAPSID;                                                
%   59|  1afv-A  3.2  3.6   86   151   15 PDB  MOLECULE: HUMAN IMMUNODEFICIENCY VIRUS TYPE 1 CAPSID                 
%   60|  2gon-C  3.2  4.3   85   138   14 PDB  MOLECULE: CAPSID PROTEIN P24 (CA);                                   
%   61|  4ph3-B  3.2  3.1   64   115    8 PDB  MOLECULE: BLV CAPSID;                                                
%   62:  3ual-A  3.2 12.1   69   230    7 PDB  MOLECULE: 14-3-3 PROTEIN EPSILON;                                    
%   63:  1o9e-A  3.2 12.2   70   231   11 PDB  MOLECULE: 14-3-3-LIKE PROTEIN C;                                     
%   64:  1o9c-A  3.2 12.3   69   231   10 PDB  MOLECULE: 14-3-3-LIKE PROTEIN C;                                     
%   65:  3cu8-A  3.2 11.3   68   229    9 PDB  MOLECULE: 14-3-3 PROTEIN ZETA/DELTA;                                 
%   66:  3ubw-A  3.2 12.1   69   230    7 PDB  MOLECULE: 14-3-3 PROTEIN EPSILON;                                    
%   67:  1o9f-A  3.2 12.2   71   231    8 PDB  MOLECULE: 14-3-3-LIKE PROTEIN C;                                     
%   68:  4fj3-A  3.2 12.2   70   220    9 PDB  MOLECULE: 14-3-3 PROTEIN ZETA/DELTA;                                 
%   69|  3dph-B  3.2  2.9   63    80   10 PDB  MOLECULE: HIV-1 CAPSID PROTEIN;                                      
%   70|  3ds3-B  3.2  2.6   59    73   10 PDB  MOLECULE: HIV-1 CAPSID PROTEIN;                                      
%   71|  4ph1-A  3.2  3.2   63    73    6 PDB  MOLECULE: BLV CAPSID;                                                
%   72|  2y4z-A  3.1  3.8   68   135   10 PDB  MOLECULE: CAPSID PROTEIN P30;                                        
%   73|  5a9e-A  3.1  3.1   61   254    3 PDB  MOLECULE: DELTAMBD GAG PROTEIN;                                      
%   74:  2liu-A  3.1  2.9   58    99    7 PDB  MOLECULE: CURA;                                                      
%   75:  2o98-B  3.1  2.8   42   237   12 PDB  MOLECULE: 14-3-3-LIKE PROTEIN C;                                     
%   76:  3ph0-C  3.1  3.3   42    53   14 PDB  MOLECULE: ASCE;                                                      
%   77:  1s7e-A  3.1  2.0   40   147   10 PDB  MOLECULE: HEPATOCYTE NUCLEAR FACTOR 6;                               
%   78:  4dx7-B  3.1  4.5   65  1033    5 PDB  MOLECULE: ACRIFLAVINE RESISTANCE PROTEIN B;                          
%   79|  3ds2-B  3.1  3.2   70    84   11 PDB  MOLECULE: HIV-1 CAPSID PROTEIN;                                      
%   80|  2eia-B  3.1  2.9   61   204   11 PDB  MOLECULE: EIAV CAPSID PROTEIN P26;                                   
%   81|  4m0i-A  3.1  2.9   61    71    8 PDB  MOLECULE: HIV-1 CAPSID PROTEIN;                                      
%   82|  4u0d-L  3.1  3.9   83   191   14 PDB  MOLECULE: GAG POLYPROTEIN;                                           
%   83|  4ph3-A  3.1  3.1   65   115    8 PDB  MOLECULE: BLV CAPSID;                                                
%   84:  4wrq-A  3.1 12.3   71   220    8 PDB  MOLECULE: 14-3-3 PROTEIN ZETA/DELTA;                                 
%   85:  1o9d-A  3.1 11.4   68   230    9 PDB  MOLECULE: 14-3-3-LIKE PROTEIN C;                                     
%   86:  4n84-B  3.1 12.5   68   226    9 PDB  MOLECULE: 14-3-3 PROTEIN ZETA/DELTA;                                 
%   87:  4fl5-B  3.1 12.2   69   229    9 PDB  MOLECULE: 14-3-3 PROTEIN SIGMA;                                      
%   88|  1a43-A  3.1  2.9   61    72    8 PDB  MOLECULE: HIV-1 CAPSID;                                              
%   89:  2c63-D  3.1 10.9   63   233   10 PDB  MOLECULE: 14-3-3 PROTEIN ETA;                                        
%   90:  4n7g-A  3.1 12.4   69   229    7 PDB  MOLECULE: 14-3-3 PROTEIN ZETA/DELTA;                                 
%   91|  1baj-A  3.1  2.8   60    71    8 PDB  MOLECULE: GAG POLYPROTEIN;                                           
%   92:  3ph0-D  3.1  3.8   48    53   15 PDB  MOLECULE: ASCE;                                                      
%   93:  3uzd-A  3.1 12.2   69   229   13 PDB  MOLECULE: 14-3-3 PROTEIN GAMMA;                                      
%   94|  4ph1-C  3.1  3.1   64    79    6 PDB  MOLECULE: BLV CAPSID;                                                
%   95:  4n84-A  3.1 12.4   71   227    8 PDB  MOLECULE: 14-3-3 PROTEIN ZETA/DELTA;                                 
%   96:  3rmr-A  3.0  4.9   99   236   11 PDB  MOLECULE: AVIRULENCE PROTEIN;                                        
%   97:  2l0q-A  3.0  3.1   53    80    6 PDB  MOLECULE: ACYL CARRIER PROTEIN;                                      
%   98:  2br9-A  3.0  2.8   42   230   12 PDB  MOLECULE: 14-3-3 PROTEIN EPSILON;                                    
%   99:  2btp-A  3.0  2.8   42   248   14 PDB  MOLECULE: 14-3-3 PROTEIN TAU;                                        
%  100:  3axy-C  3.0  2.8   42   235   12 PDB  MOLECULE: PROTEIN HEADING DATE 3A;                                   
%     :
%  101|  1qrj-A  3.0  3.1   64   214    9 PDB  MOLECULE: HTLV-I CAPSID PROTEIN;                                     
%     :
%  104|  4xfx-A  3.0  4.0   82   216   15 PDB  MOLECULE: HIV-1 CAPSID PROTEIN;                                      
%  105|  4u0d-F  3.0  3.9   83   211   14 PDB  MOLECULE: GAG POLYPROTEIN;                                           
%  106|  2jpr-A  3.0  4.4   83   145   14 PDB  MOLECULE: GAG-POL POLYPROTEIN;                                       
%  107|  4xfz-A  3.0  3.1   79   213   15 PDB  MOLECULE: HIV-1 CAPSID PROTEIN;                                      
%  108|  2l6e-A  3.0  3.9   62    94    8 PDB  MOLECULE: CAPSID PROTEIN P24;                                        
%  109|  4xfy-A  3.0  3.9   83   216   14 PDB  MOLECULE: HIV-1 CAPSID PROTEIN;                                      
%  110|  4u0d-G  3.0  3.9   82   212   15 PDB  MOLECULE: GAG POLYPROTEIN;                                           
%  111|  4u0d-B  3.0  4.0   83   207   14 PDB  MOLECULE: GAG POLYPROTEIN;                                           
%     :
%  141|  4cop-A  2.9  2.9   64    83    8 PDB  MOLECULE: CAPSID PROTEIN P24;                                        
%  142|  2xv6-C  2.9  2.8   55    68    5 PDB  MOLECULE: CAPSID PROTEIN P24;                                        
%  143|  4u0c-A  2.9  4.0   83   210   14 PDB  MOLECULE: CAPSID PROTEIN P24;                                        
%  144|  4u0d-H  2.9  4.2   83   204   14 PDB  MOLECULE: GAG POLYPROTEIN;                                           
%  145|  2pwo-B  2.9  5.4   83   143   14 PDB  MOLECULE: GAG-POL POLYPROTEIN (PR160GAG-POL);                        
%  146|  2pwm-E  2.9  5.7   83   145   14 PDB  MOLECULE: GAG-POL POLYPROTEIN;                                       
%  147:  3upv-A  2.9  7.1   59   125    7 PDB  MOLECULE: HEAT SHOCK PROTEIN STI1;                                   
%  148|  4u0d-I  2.9  3.9   83   196   14 PDB  MOLECULE: GAG POLYPROTEIN;                                           
%  149|  4u0d-A  2.9  4.2   83   216   14 PDB  MOLECULE: GAG POLYPROTEIN;                                           
%  150|  3p05-E  2.9  3.2   78   199   14 PDB  MOLECULE: HIV-1 CA;                                                  
%     :
%  165|  3ds4-B  2.9  3.0   63    75   10 PDB  MOLECULE: HIV-1 CAPSID PROTEIN;                                      
%  166|  3ds4-A  2.9  2.8   59    77   10 PDB  MOLECULE: HIV-1 CAPSID PROTEIN;                                      
%     :
%  179|  3ds2-A  2.9  3.4   68    84   12 PDB  MOLECULE: HIV-1 CAPSID PROTEIN;                                      
%     :
%  183|  3ds3-A  2.9  2.7   58    73   10 PDB  MOLECULE: HIV-1 CAPSID PROTEIN;                                      
%     :
%  186|  2x82-A  2.8  3.3   67   145    9 PDB  MOLECULE: CAPSID PROTEIN P24;                                        
%     :
%  200|  3ds0-A  2.8  3.5   63    82   10 PDB  MOLECULE: HIV-1 CAPSID PROTEIN;                                      
%  201|  2xv6-A  2.8  2.9   60    75   10 PDB  MOLECULE: CAPSID PROTEIN P24;                                        
%     :
%  204|  3mge-A  2.8  4.0   81   204   15 PDB  MOLECULE: CAPSID PROTEIN P24;                                        
%  205|  1ak4-C  2.8  4.7   82   145   15 PDB  MOLECULE: CYCLOPHILIN A;                                             
%  206|  2pwm-H  2.8  5.4   81   145   15 PDB  MOLECULE: GAG-POL POLYPROTEIN;                                       
%  207|  4u0d-J  2.8  4.2   83   204   14 PDB  MOLECULE: GAG POLYPROTEIN;                                           
%     :
%  208|  1m9x-G  2.8  4.7   83   146   14 PDB  MOLECULE: CYCLOPHILIN A;                                             
%  209|  1e6j-P  2.8  3.1   78   209   15 PDB  MOLECULE: IMMUNOGLOBULIN;                                            
%  210|  4ph0-E  2.8  4.3   98   193    8 PDB  MOLECULE: BLV CAPSID;                                                
%     :
%  211|  4hkc-A  2.8 12.1   69   229    9 PDB  MOLECULE: 14-3-3 PROTEIN ZETA/DELTA;                                 
%  212|  3axy-J  2.8 12.1   71   234    8 PDB  MOLECULE: PROTEIN HEADING DATE 3A;                                   
%  213|  2c74-B  2.8 12.3   70   234   10 PDB  MOLECULE: 14-3-3 PROTEIN ETA;                                        
%  214|  4o46-C  2.8 12.2   69   235   12 PDB  MOLECULE: 14-3-3 PROTEIN GAMMA;                                      
%  215|  3axy-D  2.8 12.1   71   235    8 PDB  MOLECULE: PROTEIN HEADING DATE 3A;                                   
%  216|  3ds1-A  2.8  2.7   57    81    5 PDB  MOLECULE: HIV-1 CAPSID PROTEIN;                                      
%     :
%  241|  2pwm-D  2.7  5.4   82   145   13 PDB  MOLECULE: GAG-POL POLYPROTEIN;                                       
%  242|  4u0d-K  2.7  4.0   83   204   14 PDB  MOLECULE: GAG POLYPROTEIN;                                           
%     :
%  244|  2kod-A  2.7  3.6   65    88    5 PDB  MOLECULE: HIV-1 CA C-TERMINAL DOMAIN;                                
%     :
%  249|  2v4x-A  2.6  3.4   68   131    9 PDB  MOLECULE: CAPSID PROTEIN P27;                                        
%     :
%  272|  3j34-H  2.6  3.6   82   231   15 PDB  MOLECULE: CAPSID PROTEIN;                                            
%     :
%  274|  2jyl-A  2.6  6.8   64    84    9 PDB  MOLECULE: CAPSID PROTEIN P24 (CA);                                   
%     :
%  363|  3j34-d  2.4  3.9   83   231   13 PDB  MOLECULE: CAPSID PROTEIN;                                            
%     :
%  409|  3j34-6  2.3  4.4   84   231   14 PDB  MOLECULE: CAPSID PROTEIN;                                            
%  410|  3j34-Z  2.3  4.3   84   231   13 PDB  MOLECULE: CAPSID PROTEIN;                                            
%     :
%  471|  4e91-A  2.2  3.0   75   131   15 PDB  MOLECULE: GAG PROTEIN;                                               
%  472|  3j34-c  2.2  4.4   82   231   16 PDB  MOLECULE: CAPSID PROTEIN;                                            
%  473|  3j34-E  2.2  4.9   84   231   15 PDB  MOLECULE: CAPSID PROTEIN;                                            
%  474|  3j34-A  2.2  3.6   83   231   16 PDB  MOLECULE: CAPSID PROTEIN;                                            
%     :
%  548|  1u7k-A  2.1  4.9   79   131    8 PDB  MOLECULE: GAG POLYPROTEIN;                                           
%  549|  4ph0-F  2.1  4.5   97   199    8 PDB  MOLECULE: BLV CAPSID;                                                
%  550|  3j34-i  2.1  4.9   81   231   16 PDB  MOLECULE: CAPSID PROTEIN;                                            
% 
The DAIL search strongly suggested that the Foamy virus structure shares some similarity with the
capsid structure of the ortho-viruses.   However, the distribution of matches across the N and
C terminal domains are mixed.   For example; taking the top 12 matches, the N-terminal domain of the Foamy
structure aligns with 7 C-terminal domains compared to 4 N-terminal domains of the ortho virsuses
and the best match with the corresponding Foamy C-terminal domain aligns with an ortho N-terminal domain.
%
% Ortho virus: big N, small C
% Foamy smaller N, bigger C
% 
%                  :         :         :         :         |         :         :         :         :         100       :         :         :         :         |         :
% Full    PIGTVIPIQHIRSVTGEPPRNPREIPIWLGRNAPAIDGVFPVTTPDLRCRIINAILGGNIGLSLTPGDCLTWDSAVATLFIRTHGTFPMHQLGNVIKGIVDQEGVATAYTLGMMLSGQNYQLVSGIIRGYLPGQAVVTALQQRLDQEIDNQTRAETFI
% 3g29A-C ---------PWAD--IMQGPS--SFVDFANRLIKAVEGS---ARAPVIIDCFRQKSQPQQLI-------TTPGEIIKYVLDRQ---------------------------------------------------------------------------
% 3g0vA-C ---------PWAD--IMQGPS--SFVDFANRLIKAVEGSAL-ARAPVIIDCFRQKSQPQQLI-------TTPGEIIKYVLDRQ---------------------------------------------------------------------------
% 4ph2A-N RHRAW-ELQDIKK--EIEN----APGSVWIQTLRLAILQADP-TPADLEQLCQYIASPQTAHM-------YQNLWLQAWK-NLPT-------------------------------------------------------------------------
% 3g29B-C ---------PWAD--IMQGPS--SFVDFANRLIKAVEGSNL-ARAPVIIDCFRQKSQPQQLI-------TTPGEIIKYVLDRQ---------------------------------------------------------------------------
% 3g1iB-C ---------PWAD--IMQGPS--SFVDFANRLIKAVEGSDL-ARAPVIIDCFRQKSQPQQLI-------TTPGEIIKYVLDRQ---------------------------------------------------------------------------
% 3g21A-C ---------PWAD--IMQGPS--SFVDFANRLIKAVEGS---ARAPVIIDCFRQKSQPQQLI-------TTPGEIIKYVLDRQ---------------------------------------------------------------------------
% 4ph0C-N RHRAW-ELQDIKKEIENKA-----PGSQWIQTLRLAILQADP-TPADLEQLCQYIASPQTAHM-------YQNLWLQAWKNLPT----------------LQISLADNL----------PDGV--PKEPII-DSLS----------------------
% 4ph0D-N RHRAW-ELQDIKKE-IENK----APGSVWIQTLRLAILQADP-TPADLEQLCQYIASPQTAHM-------YQNLWLQAWKNLP-----------------QISL-ADNL----------PDGV--PKEPIISLSY-----------------------
% 4ph2B-N RHRAW-ELQDIKK--EIEN----APGSVWIQTLRLAILQADP-TPADLEQLCQYIASPQTAHM-------YQNLWLQAWK-NLPT-------------------------------------------------------------------------
% 4cocB-C ---------SILD--IRQGPK--PFRDYVDRFLKTLRAE---VKNWMTETLLVQNANPKTIL-LGPGA-----TLEEMMTACQGVG------------------------------------------------------------------------
% 3g1gB-C ---------PWAD--IMQGPS--SFVDFANRLIKAVEGSDL-ARAPVIIDCFRQKSQPQQLI-------TTPGEIIKYVLDR----------------------------------------------------------------------------
% 1afvB-N ----------------------------------------------------------------------IVQNL-------------SPRTLNAWVKVVEEK-VIPMFSALSE--GATPQDLNTMLNTVGGHQAAMQMLKETINEEASDIAYKRWIILG
% +
% 3nteA-C      YSPTSILD--IRQGPK EPFRDYVDRFYKTLRAEQASQEVKNWMTETLLVQNANPDCKTILKALGPAATLEEMMTACQGV
% 
% PIVQNIQGMVHQAISPRTLNAWVKVVEEKAFSPEVIPMFSALSEGATPQDLNTMLNTVGGHQAAMQMLKETINEEPRGSDIAGTTSTLQEQIGWMTNNPPIPVGEIYKRWIILGLNKIVRM
% ::        : :   : :    :  : :: :   :          :: ::           : :  :   :          :                                 `
% PIISEGNRNRHRAWALRELQDIKKEIENKAPSQVWIQTLRLAIADPTPADLEQLCQYIASQTAHMTSLTAAIAAAEAANTLQGFNPQNGTLTQQSAQPNAGDLRSQYQNLWLQAWKNLPTR
% 
% 
% 4PH2A-N PIISEGN-RNRHRAWALRELQDIKKEIENKAPGSQVWIQTLRLAILQADPTPADLEQLCQYIASPVDQTAHMTSLTAAIAAAEAANTLQGFNPQNGTLTQQSAQPNAGDLRSQYQNLWLQAWKNLPTR
% i                  |                                      || ||  
% 3nteA-N PIVQNIQGQMVHQAISPRTLNAWVKVVEEKAF-SPEVIP--MFSALSEGATPQDLNTMLNTVGGHQAAMQMLKETINEEAAEWDRVHPVHAGPIAPGQMREPRGSDIAGTTSTLQEQIGWMTNNPPIPVGEIYKRWIILGLNKIVRM
%      -C YSPTSILDIRQGPKEPFRDYVDRFYKTLRAEQASQEVKNWMTETLLVQNANPDCKTILKALGPAATLEEMMTACQGV
%

\subsubsection{Domain scans}

To clarify the domain match specificity, the two domains of the Foamy virus (as defined by Taylor) 
were scanned separately using the DALI program.   The results of these scans strengthened the identification
of the relationship to the ortho capsids (Fig.?) and confirmed a reverse specificity for the N-terminal
match of the Foamy structure with the C-terminal match of the Ortho virus and {\em vica versa}, with all
top 12 hits of each domain matching their opposed counterpart.
%
% >3nte
%    Nter PIVQNIQGQMVHQAI
%         SPRTLNAWVKVVEEKAFSPEVIPMFSALSEGATPQDLNTMLNTVGGHQAAMQMLKETINEEAAEWDRVHPVHAGPIAPGQMREPRGSDIAGTTSTLQEQIGWMTNNPPIPVGEIYKRWIILGLNKIVRM
%    Cter YSPTSILDIRQGPKEPFRDYVDRFYKTLRAEQASQEVKNWMTETLLVQNANPDCKTILKALGPAATLEEMMTACQGV
% wtaylor@wt:~/ianpdbs/dali$ cut -c 6-175 newN.aln | head -14
%                  :         :         :         :         |         :         :         :        
% Nter    PIGTVIPIQHIRSVTGEPPRNPREIPIWLGRNAPAIDGVFPVTTPDLRCRIINAILGGNIGLSLTPGDCLTWDSAVATLFIRTHGTFP
% 3g1gA   ---------PWAD--IMQGPS--SFVDFANRLIKAVEGSDL-ARAPVIIDCFRQKSQPQQLI--PSTL-TTPGEIIKYVLDRQK----
% 3tirA   ---------PWAD--IMQGPS--SFVDFANRLIKAVEGSDL-ARAPVIIDCFRQKSQPQQLI-------TTPGEIIKYVLDRQ-----
% 3g1iA   ---------PWAD--IMQGPS--SFVDFANRLIKAVEGSDL-ARAPVIIDCFRQKSQPQQLI----TLTT-PGEIIKYVLDRQ-----
% 3g29A   ---------PWAD--IMQGPS--SFVDFANRLIKAVEGS---ARAPVIIDCFRQKSQPQQLI-------TTPGEIIKYVLDRQ-----
% 3g0vA   ---------PWAD--IMQGPS--SFVDFANRLIKAVEGSAL-ARAPVIIDCFRQKSQPQQLI-------TTPGEIIKYVLDRQ-----
% 3g29B   ---------PWAD--IMQGPS--SFVDFANRLIKAVEGSNL-ARAPVIIDCFRQKSQPQQLI-------TTPGEIIKYVLDRQ-----
% 3g1iB   ---------PWAD--IMQGPS--SFVDFANRLIKAVEGSDL-ARAPVIIDCFRQKSQPQQLI-------TTPGEIIKYVLDRQ-----
% 3g26A   ---------PWAD--IMQGPS--SFVDFANRLIKAVEGS---CRAPVIIDCFRQKSQPQQLI-------TTPGEIIKYVLDRQ-----
% 3dtjC   ---------SILD--IRQGPK--EPFRDYVDRFYKTLR--VKNW--MTATLLVQNANPD-TILKGPGA--TLEEMMTA-CQGV-----
% 3dtjB   ---------SILD--IRQGPK--EPFRDYVDRFYKTLR--VKNW--MTATLLVQNANPD-TILKGPGA--TLEEMMTA-CQGV-----
% 3dtjA   ---------SILD--IRQGPK--EPFRDYVDRFYKTLR--VKNW--MTATLLVQNANPD-TILKGPGA--TLEEMMTA-CQGV-----
% 3g21A   ---------PWAD--IMQGPS--SFVDFANRLIKAVEGSDL-ARAPVIIDCFRQKSQPQQLI-------TTPGEIIKYVLDRQ-----
% +
% 3nteA-C      YSPTSILD--IRQGPK--EPFRDYVDRFYKTLRaeqasqeVKNWMTETLLVQNANPDckTILKALGPAaTLEEMMTACQGV
% 
%      -N PIVQNIQGQMVHQAISPRTLNAWVKVVEEKAFSPEVIPMFSALSEGATPQDLNTMLNTVGGHQAAMQMLKETINEEAAEWDRVHPVHAGPIAPGQMREPRGSDIAGTTSTLQEQIGWMTNNPPIPVGEIYKRWIILGLNKIVRM
% 3nteA-C YSPTSILDIRQGPKEPFRDYVDRFYKTLRAEQASQEVKNWMTETLLVQNANPDCKTILKALGPAATLEEMMTACQGV
% 
% wtaylor@wt:~/ianpdbs/dali$ cut -c 6-175 newC.aln | head -14
%                  :         :         :         :         |         :         :         :         :  
% Cter    MHQLGNVIKGIVDQEGVATAYTLGMMLSGQNYQLVSGIIRGYLPGQAVVTALQQRLDQEIDNQTRAETFIQHLNAVYEILGLNARGQSIRL
% 1l6nA   SPRTLNAWVKVVEEKA-IPMFSALSE---GATPDLNTMLNTVGGHQAAMQMLKETINEEA--EIYKRWIILGLNKIVRMYS------PTSI
% 3j34U   SPRTLNAWVKVVEEKA-IPMFSALSE--GATPQDLNTMLNTVGGHQAAMQMLKETINEEA--EIYKRWIILGLNKIVRMY-------SPTS
% 4u0bF   SPRTLNAWVKVVEEKA-IPMFSALSC--GATPQDLNTMLNTVGGHQAAMQMLKETINEEA--EIYKRWIILGLNKIVRMY-------SPTS
% 4u0bG   SPRTLNAWVKVVEEK--IPMFSALSC--GATPQDLNTMLNTVGGHQAAMQMLKETINEEA--EIYKRWIILGLNKIVRMY-------SPTS
% 3h4eB   SPRTLNAWVKVVEEK--IPMFSALSC--GATPQDLNTMLNTVGGHQAAMQMLKETINEEA--EIYKRWIILGLNKIVRMY-------SPTS
% 2jprA   SPRTLNAWVKVVEEKA-IPMFSALSE--GATPQDLNTMLNTVGGHQAAMQMLKETINEEA--EIYKRWIILGLNKIVRMY-----------
% 1afvB   SPRTLNAWVKVVEEKAVIPMFSALSE--GATPQDLNTMLNTVGGHQAAMQMLKETINEEA--EIYKRWIILGLNKIVRMY-------SPTS
% 4u0bE   SPRTLNAWVKVVEEK--IPMFSALSC--GATPQDLNTMLNTV-GHQAAMQMLKETINEEA--EIYKRWIILGLNKIVRMY-------SPTS
% 4u0bK   SPRTLNAWVKVVEEK--IPMFSALSC--GATPQDLNTMLNTVGGHQAAMQMLKETINEEA--EIYKRWIILGLNKIVRMY-------SPTS
% 4u0bH   SPRTLNAWVKVVEEK--IPMFSALSC--GATPQDLNTMLNTVGGHQAAMQMLKETINEEA--EIYKRWIILGLNKIVRMY-------SPTS
% 2gonA   SPRTLNAWVKVVEEK-VIPXFSALSE--GATPQDLNTXLNTVGGHQAAXQXLKETINEEA--EIYKRWIILGLNKIVRXYS----------
% 1afvA   SPRTLNAWVKVVEEKAVIPMFSALSE--GATPQDLNTMLNTVGGHQAAMQMLKETINEEA--EIYKRWIILGLNKIVRMY-------SPTS
% +
% 3nteA-N SPRTLNAWVKVVEEKAFSPEVIPMFSALSEGATPQDLNTMLNTVGGHQAAMQMLKETINEEAAEWDRVHPVHAGPIAPGQMREPRGSDIAGTTSTLQEQIGWMTNNPPIPVGEIYKRWIILGLNKIVRM
% 
% 3nteA-N PIVQNIQGQMVHQAI
%         SPRTLNAWVKVVEEKAFSPEVIPMFSALSEGATPQDLNTMLNTVGGHQAAMQMLKETINEEAAEWDRVHPVHAGPIAPGQMREPRGSDIAGTTSTLQEQIGWMTNNPPIPVGEIYKRWIILGLNKIVRM
%      -C YSPTSILDIRQGPKEPFRDYVDRFYKTLRAEQASQEVKNWMTETLLVQNANPDCKTILKALGPAATLEEMMTACQGV
% 
The structural superpositions of each domain based on this equivalence are shown in Fig.?

Although domain transposition is not impossible in viral genomes (Ref.?),  it is sufficiently
unexpected to warant deeper investigation, especially as it is hard to imagine how an ancestral
capsid protein could tolerate such a large rearrangement and still pack to form a competent shell.

\subsection{Structural alignment significance}

\subsubsection{DALI Z-scores}

For each comparison, the DALI program calculates a Z-score, combining a estimation of significance
with protein length normalisation.   The program reports all matches over Z=2, however, when the
proteins are small and especially when the structures being compared are both predominantly
alpha-helical in nature, then matches over this cutoff includes many functionally unrelated
hits where the similarity has arisen through the fortuitous alignment of a few helices.
(For example; the top hit when scanning with the C-terminal domain is a non-capsid structure).

To calculate a stricter cutoff on score, we created a customised decoy probe by reversing the
alpha-carbon backbone then reconstructing the full atom structure (using a simple algorithm).
Fig.? plots the ranked Z-scores for the smaller Foamy N domain (red=T, cyan=F) and C domain
(orange=T, green=F)\footnote{
NB: True/false (T/F) hits were defined simply by protein descriptions that contained the
words "CAPSID", "GAG" or "P24".   This may have misclassified a few (low scoring) hits to the matrix protein
and missed some hits where the primary description referrs to a cyclophilin in complex with the capsid. 
}.   As would be expected, the larger C-term domain has hits with a higher significance than the
smaller N-term domain:  the former covers the range Z=5 to Z=3 over the true hits wheras the
latter tracks a slmilar profile running one Z-value unit lower (4--2).

The equivalent scans with the reversed domain structures (which should have no particular
relationship to the capsid or any other natural protein) also found hits with high Z-scores,
which when ranked with the native domains, had a profile that tracked closely, or just above
the N-terminal native domain.
 

\subsubsection{Customised decoys}

The mixed results observed when the original scan was made with the full-length chain showed that
the matches of each domain had comparable degrees of similarity.    To investigate if a significant
difference could be detected, we employed a method based on the generation of a population of
'decoy' models to provide a background distribution of unrelated protein scores \cite{Taylor}.
This method has the advantage that each comparison that constitutes the random pool is between
two models of the same size and secondary structure composition as the pair of native structures
being investigated.
For this study we collected 10 capsid N-terminal domains and 7 C-terminal domains, each of which 
were compared with the foamy N-terminal domain and the foamy C-terminal domain.
(Details of the structures are included in Table ?).

%
%  Run sapit.csh for each domain against the foamy N and C
%
%wtaylor@wt:~/ianpdbs/sapit$ cat rundom.csh
%# 1 = domain
%
%mkdir $argv[1]
%cd $argv[1]
%ln -s ~/util
%ln -s ~/sapit main
%ln -s .. home
%ln -s ../pdb
%tcsh main/sapit.csh $argv[1] foamN > $argv[1]N.log
%tcsh main/sapit.csh $argv[1] foamC > $argv[1]C.log
%
% and do this for each capsid domain
%
%wtaylor@wt:~/ianpdbs/sapit$ cat rundoms.csh
%foreach dom (`cat [NC]term.list`)
%	echo $dom
%	tcsh rundom.csh $dom
%end
%
% convert the comparisons into Z-scores using sapit/fits1.c
% fits1 reads the true and the random scores and discards
% random pairs outside +/-10 of the true alignment length.
% It then converts the RMS:Length (r:x) values into... 
% a[i] = r/(sqrt(x)*(1-exp(-x*x/damp)));
% The maan and std. of a[] give the Z-score for the true.
%
% NB the values of margin = 15 and damp = 1 were best
% (ie bigest mean Z and smallest std.)
%
%wtaylor@wt:~/ianpdbs/sapit$ more scoreZ.csh
%rm score*.dat
%set foamN = `echo foamN`
%set foamC = `echo foamC`
%foreach dom (`cat Nterm.list`)
%	echo $dom
%	set score = `main/fits1 $dom/$dom$foamN` 
%	echo $dom+N $score >> scoreNN.dat
%	set score = `main/fits1 $dom/$dom$foamC` 
%	echo $dom+C $score >> scoreNC.dat
%end
%foreach dom (`cat Cterm.list`)
%	echo $dom
%	set score = `main/fits1 $dom/$dom$foamN` 
%	echo $dom+N $score >> scoreCN.dat
%	set score = `main/fits1 $dom/$dom$foamC` 
%	echo $dom+C $score >> scoreCC.dat
%end
%

The degree of similarity between the domains ranged from less than 2 sigma (Z-score)
to over 5 sigma, with the latter (highly significant) result being obtained for both
a reversed (NC) and forward (CC) matching.   However, of the top 10 scores, only
three came from reversed pairings.  (Table ?).  To obtain a more quantitative consensus
for the amino/amino (NN) and carboxy/carboxy (CC) versus the reversed domain pairings
(NC and CN), the raw results were combined for each pairing, giving now not just a
single value compared to a distribution but two distrubutions.   For these data,
a significance was calculated using Student's T-test, the values of which are given
in Table ?\footnote{
The values quoted are for a two-tailed T-test, however, as it is expected that the native
comparisons should be more similar than comparisons between random models, then a
one-tailed T-test would be valid, which gives half the probability.   As the values
in the Tables are so significant and only the relative relationships are important,
then the choice is unimportant.
}.
From this it can be seen that all the four possible pairings are
highly significant with probabilities ranging from $10^{10}$ to over $10^{20}$.
It is also clear that the two reversed pairings (NC and CN) have lower probabilities
than the forward pairings (NN and CC).   Combining the probabilities ($P$) as:
$\Delta P = \log{_10}(P_{NN} P_{CC}) - log_{10}(P_{NC} P_{CN})$,
gives a value of 18 (42.7 - 25.0 = 17.7) which means that the reversed pairing is
a billion,billion times less likely than the forward pairing.  This suggests that
the reversed pairing, which was indicated originally by the Dali results, seems unlikely. 
The preferred, and biologically more reasonable, result is that the ortho virus
domain are related to the foamy virus domains as a result of genetic divergence from
a common, double domain anscestor. 

%wtaylor@wt:~/ianpdbs/sapit$ more score*.dat
%::::::::::::::
%scoreCC.dat
%::::::::::::::
%BLV6_C+C damp = 30.000000 margin = 10 n = 212 mean = 1.345119 stdv = 0.157218 y = 0.708869 mean-y = 0.636250 z = 4.046926
%BLV_C+C damp = 30.000000 margin = 10 n = 204 mean = 1.305845 stdv = 0.185196 y = 0.556341 mean-y = 0.749504 z = 4.047075
%HIV1_C+C damp = 30.000000 margin = 10 n = 174 mean = 1.291238 stdv = 0.174192 y = 0.705527 mean-y = 0.585711 z = 3.362449
%HIV6_C+C damp = 30.000000 margin = 10 n = 177 mean = 1.419853 stdv = 0.178084 y = 0.639918 mean-y = 0.779935 z = 4.379582
%HML2_C+C damp = 30.000000 margin = 10 n = 184 mean = 1.291423 stdv = 0.157803 y = 0.676068 mean-y = 0.615355 z = 3.899511
%HTLV_C+C damp = 30.000000 margin = 10 n = 163 mean = 1.259457 stdv = 0.201546 y = 0.693736 mean-y = 0.565720 z = 2.806904
%RSV_C+C damp = 30.000000 margin = 10 n = 235 mean = 1.300217 stdv = 0.179102 y = 0.403158 mean-y = 0.897059 z = 5.008660
%::::::::::::::
%scoreCN.dat
%::::::::::::::
%BLV6_C+N damp = 30.000000 margin = 10 n = 144 mean = 1.269846 stdv = 0.167814 y = 0.763161 mean-y = 0.506686 z = 3.019336
%BLV_C+N damp = 30.000000 margin = 10 n = 154 mean = 1.268382 stdv = 0.202504 y = 0.579920 mean-y = 0.688462 z = 3.399738
%HIV1_C+N damp = 30.000000 margin = 10 n = 157 mean = 1.229855 stdv = 0.169134 y = 0.593804 mean-y = 0.636051 z = 3.760632
%HIV6_C+N damp = 30.000000 margin = 10 n = 179 mean = 1.314433 stdv = 0.168393 y = 0.779728 mean-y = 0.534705 z = 3.175344
%HML2_C+N damp = 30.000000 margin = 10 n = 185 mean = 1.269326 stdv = 0.177262 y = 0.732750 mean-y = 0.536576 z = 3.027011
%HTLV_C+N damp = 30.000000 margin = 10 n = 156 mean = 1.267269 stdv = 0.151450 y = 0.684654 mean-y = 0.582614 z = 3.846900
%RSV_C+N damp = 30.000000 margin = 10 n = 155 mean = 1.226569 stdv = 0.207357 y = 0.448186 mean-y = 0.778383 z = 3.753835
%::::::::::::::
%scoreNC.dat
%::::::::::::::
%BLV_N+C damp = 30.000000 margin = 10 n = 184 mean = 1.234132 stdv = 0.227439 y = 0.399576 mean-y = 0.834556 z = 3.669370
%HIV1_N+C damp = 30.000000 margin = 10 n = 213 mean = 1.304811 stdv = 0.244515 y = 0.402160 mean-y = 0.902651 z = 3.691602
%HML2_N+C damp = 30.000000 margin = 10 n = 196 mean = 1.337005 stdv = 0.195668 y = 0.438086 mean-y = 0.898919 z = 4.594095
%HTLV_N+C damp = 30.000000 margin = 10 n = 328 mean = 1.283603 stdv = 0.205965 y = 0.457005 mean-y = 0.826598 z = 4.013291
%JSRV_N+C damp = 30.000000 margin = 10 n = 190 mean = 1.284324 stdv = 0.210871 y = 0.601744 mean-y = 0.682580 z = 3.236956
%MLV_N+C damp = 30.000000 margin = 10 n = 188 mean = 1.240526 stdv = 0.232555 y = 0.507666 mean-y = 0.732859 z = 3.151332
%MPMV_N+C damp = 30.000000 margin = 10 n = 185 mean = 1.204718 stdv = 0.233608 y = 0.522996 mean-y = 0.681723 z = 2.918237
%PSIV_N+C damp = 30.000000 margin = 10 n = 235 mean = 1.346684 stdv = 0.194710 y = 0.369476 mean-y = 0.977208 z = 5.018795
%RELIK_N+C damp = 30.000000 margin = 10 n = 237 mean = 1.358031 stdv = 0.199553 y = 0.700138 mean-y = 0.657893 z = 3.296824
%RSV_N+C damp = 30.000000 margin = 10 n = 239 mean = 1.285199 stdv = 0.214446 y = 0.525590 mean-y = 0.759609 z = 3.542199
%::::::::::::::
%scoreNN.dat
%::::::::::::::
%BLV_N+N damp = 30.000000 margin = 10 n = 251 mean = 1.315207 stdv = 0.170234 y = 0.550254 mean-y = 0.764954 z = 4.493551
%HIV1_N+N damp = 30.000000 margin = 10 n = 312 mean = 1.387546 stdv = 0.219740 y = 0.573903 mean-y = 0.813643 z = 3.702746
%HML2_N+N damp = 30.000000 margin = 10 n = 264 mean = 1.265025 stdv = 0.225198 y = 0.777153 mean-y = 0.487873 z = 2.166418
%HTLV_N+N damp = 30.000000 margin = 10 n = 400 mean = 1.324345 stdv = 0.181467 y = 0.592956 mean-y = 0.731389 z = 4.030429
%JSRV_N+N damp = 30.000000 margin = 10 n = 225 mean = 1.243907 stdv = 0.201215 y = 1.063649 mean-y = 0.180259 z = 0.895854
%MLV_N+N damp = 30.000000 margin = 10 n = 326 mean = 1.312253 stdv = 0.184224 y = 0.751373 mean-y = 0.560880 z = 3.044558
%MPMV_N+N damp = 30.000000 margin = 10 n = 269 mean = 1.304454 stdv = 0.189464 y = 0.565209 mean-y = 0.739246 z = 3.901779
%PSIV_N+N damp = 30.000000 margin = 10 n = 285 mean = 1.341958 stdv = 0.193299 y = 0.620753 mean-y = 0.721205 z = 3.731022
%RELIK_N+N damp = 30.000000 margin = 10 n = 234 mean = 1.377677 stdv = 0.200315 y = 0.639005 mean-y = 0.738672 z = 3.687560
%RSV_N+N damp = 30.000000 margin = 10 n = 204 mean = 1.299841 stdv = 0.242518 y = 0.542528 mean-y = 0.757312 z = 3.122707
%
%scoreNC.dat:PSIV_N+C damp = 30.000000 margin = 10 n = 235 mean = 1.346684 stdv = 0.194710 y = 0.369476 mean-y = 0.977208 z = 5.018795
%scoreCC.dat:RSV_C+C damp = 30.000000 margin = 10 n = 235 mean = 1.300217 stdv = 0.179102 y = 0.403158 mean-y = 0.897059 z = 5.008660
%scoreNC.dat:HML2_N+C damp = 30.000000 margin = 10 n = 196 mean = 1.337005 stdv = 0.195668 y = 0.438086 mean-y = 0.898919 z = 4.594095
%scoreNN.dat:BLV_N+N damp = 30.000000 margin = 10 n = 251 mean = 1.315207 stdv = 0.170234 y = 0.550254 mean-y = 0.764954 z = 4.493551
%scoreCC.dat:HIV6_C+C damp = 30.000000 margin = 10 n = 177 mean = 1.419853 stdv = 0.178084 y = 0.639918 mean-y = 0.779935 z = 4.379582
%scoreCC.dat:BLV_C+C damp = 30.000000 margin = 10 n = 204 mean = 1.305845 stdv = 0.185196 y = 0.556341 mean-y = 0.749504 z = 4.047075
%scoreCC.dat:BLV6_C+C damp = 30.000000 margin = 10 n = 212 mean = 1.345119 stdv = 0.157218 y = 0.708869 mean-y = 0.636250 z = 4.046926
%scoreNN.dat:HTLV_N+N damp = 30.000000 margin = 10 n = 400 mean = 1.324345 stdv = 0.181467 y = 0.592956 mean-y = 0.731389 z = 4.030429
%scoreNC.dat:HTLV_N+C damp = 30.000000 margin = 10 n = 328 mean = 1.283603 stdv = 0.205965 y = 0.457005 mean-y = 0.826598 z = 4.013291
%scoreNN.dat:MPMV_N+N damp = 30.000000 margin = 10 n = 269 mean = 1.304454 stdv = 0.189464 y = 0.565209 mean-y = 0.739246 z = 3.901779
%
%wtaylor@wt:~/ianpdbs/sapit$ tcsh scoreZ.csh 1 15
%BLV_N
%HIV1_N
%HML2_N
%HTLV_N
%JSRV_N
%MLV_N
%MPMV_N
%PSIV_N
%RELIK_N
%RSV_N
%BLV6_C
%BLV_C
%HIV1_C
%HIV6_C
%HML2_C
%HTLV_C
%RSV_C
%
%stdev.= 0.18877
%Zscore= 3.30819
%T-tests
%
%N+foamN
% in1 = 10 maln = 72 naln = 80 mean1 = 0.666767 stdv1 = 0.160854 in2 = 833 mean2 = 1.316649 stdv2 = 0.211645
%Avg: 6.67e-01 < 1.32e+00 Tprob=4.62e-21 **
%StD: 1.61e-01 = 2.12e-01 Fprob=1.84e-01 
%N+foamC
% in1 = 10 maln = 79 naln = 87 mean1 = 0.492081 stdv1 = 0.101925 in2 = 996 mean2 = 1.290236 stdv2 = 0.220791
%Avg: 4.92e-01 < 1.29e+00 Tprob=4.09e-10 **
%StD: 1.02e-01 < 2.21e-01 Fprob=7.37e-03 **
%C+foamN
% in1 = 7 maln = 59 naln = 74 mean1 = 0.650558 stdv1 = 0.117477 in2 = 992 mean2 = 1.247814 stdv2 = 0.189033
%Avg: 6.51e-01 < 1.25e+00 Tprob=2.35e-16 **
%StD: 1.17e-01 = 1.89e-01 Fprob=1.12e-01 
%C+foamC
% in1 = 7 maln = 60 naln = 80 mean1 = 0.622420 stdv1 = 0.111796 in2 = 1189 mean2 = 1.300281 stdv2 = 0.176998
%Avg: 6.22e-01 < 1.30e+00 Tprob=3.81e-23 **
%StD: 1.12e-01 = 1.77e-01 Fprob=1.20e-01 
%
% combining
%
%wtaylor@wt:~/ianpdbs/sapit$ echo 4.62e-21 3.83e-23 | awk '{print $1,$2, $1*$2, -log($1*$2)}'
%4.62e-21 3.83e-23 1.76946e-43 98.4405
%wtaylor@wt:~/ianpdbs/sapit$ echo 4.09e-10 2.35e-16 | awk '{print $1,$2, $1*$2, -log($1*$2)}'
%4.09e-10 2.35e-16 9.6115e-26 57.6043
%
%wtaylor@wt:~/ianpdbs/sapit$ echo 4.62e-21 3.83e-23 | awk '{print $1,$2, $1*$2, -log($1*$2)/log(10)}'
%4.62e-21 3.83e-23 1.76946e-43 42.7522
%wtaylor@wt:~/ianpdbs/sapit$ echo 4.09e-10 2.35e-16 | awk '{print $1,$2, $1*$2, -log($1*$2)/log(10)}'
%4.09e-10 2.35e-16 9.6115e-26 25.0172
%
% 42.7 - 25.0 = 17.7, 10**18 = 10**9 * 10**9 = billion,billion 
%
The reversed pairing, nevertheless, still has high structural significance and this
suggests that the two domains are derived from a prior gene-duplication event that has
been retained more clearly in the foamy viruses.   Comparing the two foamy domins gives
a Z-score of 2.077 sigma which, although of marginal significance, supports this model.
A similar comparison in structures of the ortho virsuses with both domain gives a
similar picture on an individual basis, with Z-scores ranging from 2 to 4.  However,
as with the comparisons with the foamy virus, these can be pooled to allow a joint
T-test to be applied.   This cave a probability of $1/10^8$ that the true N/C domain
comparisons were drawn from the random distribution, adding strong support to the
hypothesis of an ancient gene duplication occuring before the split of the ortho 
and foamy virus families.

%
% foamyN vs foamyC
%
%wtaylor@wt:~/ianpdbs/sapit$ main/fits1 foamN+foamC 30 10
% damp = 30.000000 margin = 10
% n = 157 mean = 1.247504 stdv = 0.164467 y = 0.905853 mean-y = 0.341651 z = 2.077320
%
%wtaylor@wt:~/ianpdbs/sapit$ main/fits1 HIV1_N+HIV1_C 30 10
% damp = 30.000000 margin = 10
% n = 317 mean = 1.348796 stdv = 0.237078 y = 0.879874 mean-y = 0.468922 z = 1.977924
%
%wtaylor@wt:~/ianpdbs/sapit$ ~/util/sap pdb/HIV1_N pdb/HIV1_C
%
%
%
%wtaylor@wt:~/ianpdbs/sapit$ main/fits1 BLV_N+BLV_C 30 10
% damp = 30.000000 margin = 10
% n = 249 mean = 1.314609 stdv = 0.189414 y = 0.548639 mean-y = 0.765970 z = 4.043883
%wtaylor@wt:~/ianpdbs/sapit$ main/fits1 HIV1_N+HIV1_C 30 10
% damp = 30.000000 margin = 10
% n = 317 mean = 1.348796 stdv = 0.237078 y = 0.879874 mean-y = 0.468922 z = 1.977924
%wtaylor@wt:~/ianpdbs/sapit$ main/fits1 HML2_N+HML2_C 30 10
% damp = 30.000000 margin = 10
% n = 191 mean = 1.334377 stdv = 0.203411 y = 0.559553 mean-y = 0.774824 z = 3.809163
%wtaylor@wt:~/ianpdbs/sapit$ main/fits1 HTLV_N+HTLV_C 30 10
% damp = 30.000000 margin = 10
% n = 223 mean = 1.254856 stdv = 0.212137 y = 0.797973 mean-y = 0.456883 z = 2.153713
%wtaylor@wt:~/ianpdbs/sapit$ main/fits1 RSV_N+RSV_C 30 10
% damp = 30.000000 margin = 10
% n = 85 mean = 1.251280 stdv = 0.329498 y = 0.681746 mean-y = 0.569534 z = 1.728493
%wtaylor@wt:~/ianpdbs/sapit$ main/stest orthoNC 30 10
% argv[1] = orthoNC damp = 30.000000
% in1 = 6 maln = 49 naln = 78 mean1 = 0.728939 stdv1 = 0.156379 in2 = 1864 mean2 = 1.281632 stdv2 = 0.239511
%Avg: 7.29e-01 < 1.28e+00 Tprob=1.88e-08 **
%StD: 1.56e-01 = 2.40e-01 Fprob=1.69e-01 
%
