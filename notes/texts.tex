\section{Introduction}
Foamy virus capsids are of interest.

\section{Results}

\subsection{DALI searches}

\subsubsection{Full chain scan}

A scan of the full-length foamy structure using the DALI server
{\tt http://ekhidna.biocenter.helsinki.fi/dali_server}
over the 90\% non-redundant protein structure databank
identified a wide selection of retroviral capsid structures.  In the ranked list of structue hits,
capsids identified from position 2 to position 550.

The DAIL search strongly suggested that the Foamy virus structure shares some similarity with the
capsid structure of the ortho-viruses.   However, the distribution of matches across the N and
C terminal domains are mixed.   For example; taking the top 12 matches, the N-terminal domain of the Foamy
structure aligns with 7 C-terminal domains compared to 4 N-terminal domains of the ortho virsuses
and the best match with the corresponding Foamy C-terminal domain aligns with an ortho N-terminal domain.

\subsubsection{Domain scans}

To clarify the domain match specificity, the two domains of the Foamy virus (as defined by Taylor) 
were scanned separately using the DALI program.   The results of these scans strengthened the identification
of the relationship to the ortho capsids (Fig.?) and confirmed a reverse specificity for the N-terminal
match of the Foamy structure with the C-terminal match of the Ortho virus and {\em vica versa}, with all
top 12 hits of each domain matching their opposed counterpart.
The structural superpositions of each domain based on this equivalence are shown in Fig.?

Although domain transposition is not impossible in viral genomes (Ref.?),  it is sufficiently
unexpected to warant deeper investigation, especially as it is hard to imagine how an ancestral
capsid protein could tolerate such a large rearrangement and still pack to form a competent shell.

\subsection{Structural alignment significance}

\subsubsection{DALI Z-scores}

For each comparison, the DALI program calculates a Z-score, combining a estimation of significance
with protein length normalisation.   The program reports all matches over Z=2, however, when the
proteins are small and especially when the structures being compared are both predominantly
alpha-helical in nature, then matches over this cutoff includes many functionally unrelated
hits where the similarity has arisen through the fortuitous alignment of a few helices.
(For example; the top hit when scanning with the C-terminal domain is a non-capsid structure).

To calculate a stricter cutoff on score, we created a customised decoy probe by reversing the
alpha-carbon backbone then reconstructing the full atom structure (using a simple algorithm).
Fig.? plots the ranked Z-scores for the smaller Foamy N domain (red=T, cyan=F) and C domain
(orange=T, green=F)\footnote{
NB: True/false (T/F) hits were defined simply by protein descriptions that contained the
words "CAPSID", "GAG" or "P24".   This may have misclassified a few (low scoring) hits to the matrix protein
and missed some hits where the primary description referrs to a cyclophilin in complex with the capsid. 
}.   As would be expected, the larger C-term domain has hits with a higher significance than the
smaller N-term domain:  the former covers the range Z=5 to Z=3 over the true hits wheras the
latter tracks a slmilar profile running one Z-value unit lower (4--2).

The equivalent scans with the reversed domain structures (which should have no particular
relationship to the capsid or any other natural protein) also found hits with high Z-scores,
which when ranked with the native domains, had a profile that tracked closely, or just above
the N-terminal native domain.
 

\subsubsection{Customised decoys}

The mixed results observed when the original scan was made with the full-length chain showed that
the matches of each domain had comparable degrees of similarity.    To investigate if a significant
difference could be detected, we employed a method based on the generation of a population of
'decoy' models to provide a background distribution of unrelated protein scores \cite{Taylor}.
This method has the advantage that each comparison that constitutes the random pool is between
two models of the same size and secondary structure composition as the pair of native structures
being investigated.
For this study we collected 10 capsid N-terminal domains and 7 C-terminal domains, each of which 
were compared with the foamy N-terminal domain and the foamy C-terminal domain.
(Details of the structures are included in Table ?).


The degree of similarity between the domains ranged from less than 2 sigma (Z-score)
to over 5 sigma, with the latter (highly significant) result being obtained for both
a reversed (NC) and forward (CC) matching.   However, of the top 10 scores, only
three came from reversed pairings.  (Table ?).  To obtain a more quantitative consensus
for the amino/amino (NN) and carboxy/carboxy (CC) versus the reversed domain pairings
(NC and CN), the raw results were combined for each pairing, giving now not just a
single value compared to a distribution but two distrubutions.   For these data,
a significance was calculated using Student's T-test, the values of which are given
in Table ?.   From this it can be seen that all the four possible pairings are
highly significant with probabilities ranging from $10^{10}$ to over $10^{20}$.
It is also clear that the two reversed pairings (NC and CN) have lower probabilities
than the forward pairings (NN and CC).   Combining the probabilities ($P$) as:
$\Delta P = \log{_10}(P_{NN} P_{CC}) - log_{10}(P_{NC} P_{CN})$,
gives a value of 18 (42.7 - 25.0 = 17.7) which means that the reversed pairing is
a billion,billion times less likely than the forward pairing.  This suggests that
the reversed pairing, which was indicated originally by the Dali results, seems unlikely. 
The preferred, and biologically more reasonable, result is that the ortho virus
domain are related to the foamy virus domains as a result of genetic divergence from
a common, double domain anscestor. 

The reversed pairing, nevertheless, still has high structural significance and this
suggests that the two domains are derived from a prior gene-duplication event that has
been retained more clearly in the foamy viruses.   Comparing the two foamy domins gives
a Z-score of 2.077 sigma which, although of marginal significance, supports this model.
A similar comparison in structures of the ortho virsuses with both domain gives a
similar picture on an individual basis, with Z-scores ranging from 2 to 4.  However,
as with the comparisons with the foamy virus, these can be pooled to allow a joint
T-test to be applied.   This cave a probability of $1/10^8$ that the true N/C domain
comparisons were drawn from the random distribution, adding strong support to the
hypothesis of an ancient gene duplication occuring before the split of the ortho 
and foamy virus families.

